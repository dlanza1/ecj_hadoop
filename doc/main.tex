\documentclass{estilos-y-libreria}

\title{Integración de una herramienta de computo evolutivo y una de procesamiento masivo de informaci\'on}

\autores{Daniel Lanza Garc\'ia}{}

\director{Francisco Fernandez de Vega}
\directora{}
\codirector{Francisco Chavez de la O}
\codirectora{}

\date{Junio 2015}

\prologo{
En las primeras l\'ineas que describen este proyecto fin de grado, explicamos como se ha realizado la integraci\'on entre dos herramientas bien conocidas por la comunidad, ECJ y Hadoop.

El hecho de mi inter\'es por tecnolog\'ias Big Data y el estar trabajando con investigadores inmersos en la computaci\'on evolutiva, provoc\'o la idea de la uni\'on de ambos campos de conocimiento. Hemos compartido muchas conversaciones donde intent\'abamos refinar la idea, eligiendo las herramientas a integrar, el modo en el que lo hacíamos, mejoras, cambios, etc. Estas conversaciones consegu\'ian mantenernos cada vez m\'as inmersos e inquietos con el proyecto.

Las estancias que hemos realizado con otros grupos de investigaci\'on, en este caso Mexicanos, o algunos de los estudiantes que han venido al centro, han aportado valor al proyecto. Las ideas propuestas por las personas con las que se ha compartido la idea han sido tenidas muy en cuenta. Deducimos fácilmente como la colaboraci\'on entre diferentes grupos o personas no puede m\'as que aportar buenas ideas a un proyecto.

Aunque no alcanzados todos los objetivos deseados, ya que no dejan de surgir ideas, los resultados obtenidos sugieren que el tiempo dedicado ha merecido la pena.
}
\agradecimientos{
Quiero dar mi mas sincero agradecimiento a Don Francisco Fern\'andez de Vega, por haberme dirigido este proyecto de fin de grado. Gracias por su paciencia y por las horas que ha dedicado a este proyecto d\'andole vuelta a ideas o simplemente pensando como podr\'ia quedar mejor la idea final. Sin olvidar su incondicional apoyo a mi trabajo, intentando siempre impulsar mi carrera tan alto como sea posible.

Tambi\'en he de agradecer a Don Francisco Ch\'avez de la O, codirector de \'este proyecto, sin su voluntad y ganas de trabajar no se hubiera podido gestar la idea final que ha surgido. Agradecer las atenciones, el tiempo que ha perdido conmigo en el despliegue de las herramientas y sobre todo por su apoyo t\'ecnico y moral durante nuestra estancia Mexicana.

A los miembros del laboratorio EvoVisi\'on de CICESE (Ensenada, M\'exico), en especial al Doctor Gustavo Olague y a algunos de sus estudiantes, Eddie y Daniel. Durante la estancia en su laboratorio fuimos especialmente acogidos, all\'i se desarroll\'o parte del proyecto y ellos tuvieron especial inter\'es proponiendo ideas.

A los miembros del instituto tecnol\'ogico de Tijuana (M\'exico), teniendo especial atenci\'on con el Doctor Leonardo Trujillo y algunos de sus alumnos, Enrique y Yuliana, que consiguieron hacer de nuestra estancia una experiencia muy placentera y productiva, mostrando su interés por el proyecto y a\~nadi\'endole valor.

A mi jefe y compa\~neros de trabajo en el CERN, los cuales han sido suficientemente flexibles para que pueda dedicarle alguna que otra hora de trabajo al proyecto, sin duda sin este tiempo no podr\'ia haberse llevado a cabo.

A C\'esar Benavides, sin el cual no podr\'ia haberse realizado la integraci\'on del proyecto de reconocimiento facial. Agradecerle el haber hecho su implementaci\'on compatible con las necesidades del proyecto y el tiempo perdido en nuestras continuas reuniones virtuales.

En general, a todos aquellos profesores y alumnos del Centro Universitario de M\'erida que de una u otra forma me han ayudado a realizar este proyecto, que aunque no les mencione de forma expl\'icita, no les puedo negar un sincero agradecimiento.

Tambi\'en agradecer a mis amigos Rafa y Carlos, los cuales me han aguant\'ado durante buena parte de la realizaci\'on del proyecto ayudando de una manera u otra en todo lo que estuviera en sus manos.

Finalmente me gustar\'ia agradecer especialmente el apoyo que me han dado mis padres, Ana y Liborio, los cuales me animan con todo lo que me propongo y como no pod\'ia ser de otra manera tambi\'en he tenido su motivaci\'on y empuje para este proyecto, preocup\'andose de su evoluci\'on en todo momento.

A todos, muchas gracias.
}

%\usepackage{wrapfig}
\usepackage[utf8]{inputenc}

\usepackage{xcolor}
\usepackage{listings}

\begin{document}

\frontmatter
\hacerportada
\hacercontraportada
\newpage{\ }
\thispagestyle{empty} 
\hacerprologo
\haceragradecimientos
\setcounter{secnumdepth}{2}
\setcounter{tocdepth}{2}
\tableofcontents
\listoffigures
\listoftables

\mainmatter

\chapter{Introducci\'on}
	\section{Motivaciones\label{motivaciones}}
		Muchos problemas computacionales de diferente naturaleza se intentan afrontar haciendo uso de modelos computacionales tradicionales obteniendo no muy buenos resultados, la computación evolutiva aporta un enfoque bioinspirado que en muchos casos consigue proporcionar resultados m\'as que aceptables. Es por esto que su popularidad est\'a en aumento, siendo aplicada esta metodolog\'ia a problemas de muy diversa \'indole.

Con el paso de los a\~nos y la evoluci\'on de los sistemas computacionales, la computaci\'on evolutiva se ha intentado aplicar a la resoluci\'on de problemas cada vez m\'as complejos, lo cual en la mayor\'ia de los casos, suele conllevar el uso de m\'as recursos como capacidad de c\'omputo, memoria y tiempo. Esto hace que se busquen soluciones para mejorar el uso de estos recursos.

Numerosas investigaciones se han llevado a cabo con el fin de minimizar el uso de uno de los recursos m\'as importantes mencionados anteriormente, el tiempo. Se han dise\~nado algoritmos en este modelo cuya ejecuci\'on puede tomar a\~nos y que por lo tanto su tiempo de ejecuci\'on es impracticable. Varias metodolog\'ias se han aplicado para reducir el tiempo de ejecuci\'on, una de ellas es la paralelizaci\'on de parte del proceso, esto puede llevarse a cabo con los procesadores de \'ultima generaci\'on los cuales poseen varios n\'ucleos de procesamiento o tambi\'en se puede conseguir con el uso de varios computadores conectados en red.

Este trabajo plantea una soluci\'on para utilizar los recursos disponibles de forma eficiente y as\'i reducir los tiempos de ejecuci\'on, la soluci\'on que se describir\'a hace uso de una herramienta que explota ambas metodolog\'ias, ejecuci\'on paralela en procesadores multi-n\'ucleo y uso de numerosas computadoras.
	\section{Objetivos}
	\section{Recursos empleados}
	\section{Organizaci\'on del documento}

\chapter{An\'alisis del sistema}
	En este cap\'itulo nos sumergiremos en los dos campos de conocimiento que este proyecto contempla, la computación evolutiva y el procesamiento masivo de información. Ambos están adquiriendo cada vez m\'as importancia en el mundo de la computación y la resolución de problemas no convencionales.

Se cubrir\'an temas como porque surgen estás metodologías, sus propósitos y de que manera intentan resolver el problema para el cual han sido desarrolladas.
	\section{Algoritmos evolutivos}
		Existen problemas computacionales que no pueden ser resueltos con técnicas tradicionales, o porque no existe una que pueda proporcionar un resultado aceptable o porque la técnica aplicada necesita de un tiempo o recursos de los que no se disponen. Así es el caso en problemas NP donde una búsqueda exhaustiva encontraría la mejor solución pero el tiempo necesario para su ejecución se hace impracticable. Para este tipo de problemas se buscan técnicas que no proporcionan siempre la mejor soluci\'on pero que intentar acercarse lo máximo posible haciendo uso de recursos razonables, a estos problemas se les conoce como problemas de optimizaci\'on.

Se han desarrollado diferentes formas de afrontar estos problemas de optimizaci\'on y una de ellas es la computación evolutiva, este modelo se basa en las teoría de la evolución que Charles Darwin postul\'o. Esta idea de aplicar la teor\'ia Darwiniana de la evolución surgió en los a\~nos 50 y desde entonces han surgidos diferentes corrientes de investigación:

\begin{itemize}
	\item Algoritmos genéticos, donde los individuos de la población son representados por cadenas de bits o números.
	\item Programación evolutiva, una variación de los algoritmos genéticos, donde lo que cambia es la representación de los individuos. En el caso de la Programación evolutiva los individuos son ternas cuyos valores representan estados de un autómata finito. 
	\item Estrategias evolutivas, se diferencia de las demás en que la representación de cada individuo de la población consta de dos tipos de variables: las variables objeto, posibles valores que hacen que la función objetivo alcance el óptimo, y las variables estratégicas, las cuales indican de qué manera las variables objeto son afectadas por la mutación.. 
\end{itemize}

Este modelo por tanto, se basa generalmente en la evolución de una población y la lucha por la supervivencia. En su teoría, Darwin dictamin\'o que durante muchas generaciones, la variación, la selección natural y la herencia dan forma a las especies con el fin de satisfacer las demandas del entorno, con la misma idea pero con el fin de satisfacer una buena solución al problema que se plantee, surge la computación evolutiva. Podemos observar entonces, algunos elementos importantes como son:

\begin{itemize}
	\item La población de individuos, donde cada una de ellos representa directa o indirectamente una solución al problema.
	\item Aptitud de los individuos, atributo que describe cuanto de cerca est\'a este individuo (solución) de la solución \'optima. 
	\item Procedimientos de sección, es la estrateg\'ia a seguir para elegir los progenitores de la siguiente genraci\'on. \'Esta normalmente elige a los individuos mas apto pero existen otras muchas técnicas.
	\item Procedimiento de transformaci\'on, se lleva a cabo sobre los individuos seleccionados y puede consistir en la combinación de varios individuos o en la mutación (cambios normalmente aleatorios en el individuo).
\end{itemize}

Para llevar a cabo la implementación en computadoras, se ha dividido el problema en diferentes fases y procedimientos, los cuales se ejecutan con un orden determinado, describimos a continuación de forma general como se lleva a cabo la resolución de problemas utilizando este modelo.

\begin{enumerate}
	\item Inicializaci\'on, en esta primera fase se genera la población inicial, normalmente se genera una cantidad de individuos que es configurada y cada uno de ellos se genera de manera aleatoria, siempre respetando las restricciones que el problema imponga a la solución.
	\item Evaluaci\'on, se calcula la aptitud de cada uno de los individuos de la población para poder determinar posteriormente cuales son m\'as aptos.
	\item Las fases que siguen a continuación se repiten hasta que se cumpla una de las siguientes dos condiciones: que se encuentre la solución \'optima o que se alcance un l\'imite impuesto por el programador como un n\'umero de generaciones máximo o un tiempo máximo.
	\begin{enumerate}
		\item Selección, siguiendo la estrategia de selección de progenitores elegida, se eligen individuos de la población. Normalmente los que sean m\'as aptos.
		\item Procreaci\'on, utilizando los individuos seleccionados, se combinan para generar nuevos individuos, y con ellos una nueva población (generación).
		\item Mutaci\'on, a un porcentaje de los individuos recién generados se les aplica una modificación aleatoria.
		\item Evaluaci\'on, todos los individuos de la nueva poblaci\'on son evaluados.
	\end{enumerate}
\end{enumerate}

Con el fin de entender mejor este proceso, se muestra una imagen \ver{fases-evolutivo} donde se observan cada una de las etapas descritas anteriormente.

\figuraSinMarco{0.7}{imagenes/fases-evolucion}{Fases del proceso evolutivo}{fases-evolutivo}{}

Varias herramientas han surgido en la comunidad para ayudar a la investigación de este modelo, en diferentes lenguajes de programación y plataformas. En nuestro caso hemos elegido ECJ que es un framework bien conocido por la comunidad, implementado en el lenguaje de programación Java y que posee una flexibilidad importante para la ejecución de problemas de muy distinta naturaleza. M\'as adelante \verapartado{desarrollo-ecj} se describe con m\'as detalle esta herramienta, explicando su funcionamiento e implementación.

\subsection{Paralelizaci\'on\label{analisis-evolutivos-paralelizacion}}

Como hemos comentado anteriormente \verapartado{motivaciones} este proyecto surge de la necesidad de optimizar el uso de recursos cuando se intentan resolver problemas complejos con este modelo. Con este fin, y con la posibilidad de paralizar el procesamiento de algunas de las partes del proceso, surge la viabilidad de este proyecto.

Varias partes del proceso evolutivo pueden ser paralizadas, pero no todas merecen el esfuerzo ya que el coste en algunas de ellas es mínimo. Una de las fases que suele conllevar un coste computacional alto y que su paralelizaci\'on en la mayoría de problemas es sencilla, es la fase de evaluaci\'on de individuos.

Se han utilizado diferentes técnicas para este prop\'osito //TODO(incluir referencias), una de ellas es la ejecución de la evaluación de individuos haciendo uso de procesadores multin\'ucleo/multihilo. Esta t\'ecnica consigue buenos resultados pero se limita a las capacidades del procesador que posea la computadora donde se ejecute. Otro intento para llevar a cabo la paralelizaci\'on del proceso ha sido la ejecución en diferentes m\'aquinas, las cuales se conectan haciendo uso de una red. Este planteamiento requiere de una implementación m\'as compleja y no suele explotar todos los recursos de forma eficiente, adema\'as de que algunas soluciones planteadas carecen de la escalabilidad deseada. Tambi\'en han surgido implementaciones que hacen uso de ambas t\'ecnicas, esta solución suele ser la m\'as apropiada ya que hace un uso m\'as eficiente del hardware disponible, aunque requiere de una implementación a\'un m\'as compleja.

Con el fin de llevar a cabo la paralelizaci\'on, han surgido diferentes modelos paralelos, algunos de los cuales se describen a continuación:

\begin{itemize}
	\item Modelo maestro-esclavo, se mantiene una población donde la evaluación del fitness y/o la aplicación de los operadores genéticos se hace en paralelo. Se implementan procesos maestro-esclavo, donde el maestro almacena la población y los esclavos solicitan parte de la población al proceso maesto para evaluarla.
	\item Modelo de Islas, la población de individuos se divide en subpoblaciones que evolucionan independientemente. Ocasionalmente, se producen migraciones entre ellas, permitiéndoles intercambiar información genética. Con la utilización de la migración, este modelo puede explotar las diferencias en las subpoblaciones; esta variación representa una fuente de diversidad genética. Sin embargo, si un número de individuos emigran en cada generación, ocurre una mezcla global y se eliminan las diferencias locales, y si la migración es infrecuente, es probable que se produzca convergencia prematura en las subpoblaciones.
	\item Modelo celular, se coloca cada individuo en una matriz, donde cada uno sólo podrá buscar reproducirse con los individuos que tenga a su alrededor escogiendo al azar o al mejor adaptado. El descendiente pasar\'a a ocupar una posición cercana. No hay islas en este modelo, pero hay efectos potenciales similares. Asumiendo que el cruce esta restringido a individuos adyacentes, dos individuos separados por 20 espacios están tan aislados como si estuvieran en dos islas, este tipo de separación es conocido como aislamiento por distancia. Luego de la primera evaluación, los individuos están todavía distribuidos al azar sobre la matriz. Posteriormente, empiezan a emerger zonas como cromosomas y adaptaciones semejantes. La reproducción y selección local crea tendencias evolutivas aisladas, luego de varias generaciones, la competencia local resultara en grupos m\'as grandes de individuos semejantes.
\end{itemize}

La implementación del paralelismo de estos y otros modelos han hecho uso de tecnologías que permitan la comunicación y distribución de los diferentes procesos que llevan a cabo el proceso evolutivo, algunas de ellas se describen a continuación:

\begin{itemize}
	\item MPI, es un estándar que define la sintaxis y la semántica de las funciones contenidas en una biblioteca de paso de mensajes diseñada para ser usada en programas que exploten la existencia de múltiples procesadores.
	\item PVM, es una biblioteca para el cómputo paralelo en un sistema distribuido de computadoras. Está diseñado para permitir que una red de computadoras heterogénea comparta sus recursos de cómputo (como el procesador y la memoria RAM) con el fin de aprovechar esto para disminuir el tiempo de ejecución de un programa al distribuir la carga de trabajo en varias computadoras.
	\item BOINC, una plataforma de computación voluntaria de propósito general para proyectos de computación distribuida, que permite compartir los recursos de las computadoras de sus contribuyentes con otros proyectos. 
\end{itemize}

El inter\'es por explorar el uso de otras tecnologías para la paralelizaci\'on de los procesos evolutivos nos ha llevado a la integración de una herramienta de c\'omputo evolutivo con una herramienta que tiene como principal propósito el procesamiento masivo de información, algo que conlleva la distribución y paralelizaci\'on de multiples tareas a lo largo de amplios clusters de computadores.





















	\section{Procesamiento masivo de informaci\'on}
		Vivimos en el momento m\'as \'algido en la genraci\'on de informaci\'on, nunca antes había existido plataformas que generaran la cantidad de información que se genera hoy. El hecho de que del análisis de grandes cantidades de información se puedan extraer valiosos datos como estrategias de negocio, hacen que numerosas empresas y organizaciones almacenen cantidades ingentes de información para poder sacarle el máximo partido.

La generaci\'on de informaci\'on se est\'a produciendo en \'ambitos muy dispares, estos pueden ser redes sociales que mantienen millones de usuarios, grandes empresas con muchos clientes, laboratorios de f\'isica con redes de millones sensores y muchos otros ejemplos que podríamos mencionar. Todos ellos queriéndole sacan el máximo valor a la información que recaban.

\subsubsection{Computaci\'on distribuida}

Cuando hablamos de cantidades de información\'on del orden de terabytes o petabytes no podemos pensar en otra cosa que no sea computación distribuida. Para una sola m\'aquina, el tiempo que conllevar\'ia procesar esas cantidades de datos podr\'ian ser del orden de a\~nos.

Por tanto, la computaci\'on distribuida es la \'unica soluci\'on con el hardware que hoy en d\'ia manejamos. Esta soluci\'on implica el uso de m\'ultiples computadores conectados a la red, cada una de las cuales tiene su propio procesador, arquitectura, etc, con lo que pueden ser totalmente heterog\'eneas, en contraposici\'on a la computaci\'on paralela, que consiste en utilizar m\'as de un hilo de procesamiento simult\'aneamente para ejecutar un \'unico programa. Idealmente, el procesamiento paralelo permite que un programa se ejecute m\'as r\'apido, en la pr\'actica, suele ser difi\'icil dividir un programa de forma que CPU separadas ejecuten diferentes porciones del programa sin ninguna interacci\'on.

\subsection{Modelo computacional: Map/Reduce}

En este \'ambito, surge la necesidad de dise\~nar herramientas que puedan no solo mantener esta informaci\'on, si no también que tengan la capacidad de analizarla y extraer el valor que se desea de una forma distribuida y con un modelo sencillo. 

\textit{Google}, el buscador de internet m\'as utilizado en el mundo, ha hecho frente ha este problema antes que nadie ya que desde hace a\~nos maneja cantidades de información realmente grandes, es por esto que sus investigaciones y experiencia son avanzadas. Varios a\~nos atrás hicieron p\'ublica \cite{paper-mapreduce} una solución para el an\'alisis de grandes cantidad de datos de forma distribuida y con un modelo que da soluci\'on a la complejidad de dividir el problema para poder paralelizarlo, ha este modelo se le conoce como Map/Reduce. Esta publicación ha dado pie a que se implemente una herramienta que se conoce con el nombre de Hadoop \cite{hadoop}, la cual se describe con m\'as detalle en un cap\'itulo posterior \verapartado{desarrollo-hadoop}. La creaci\'on de esta herramienta y el hecho de que se hayan obtenidos buenos resultados de ella, ha provocado que surjan otras muchas herramientas a su alrededor, las cuales ayudan a diferentes tareas como el volcado de información (Sqoop), bases de datos para consulta (Impala), gestion de flujos de datos (Flume) y otras muchas.

La propuesta de computación distribuida que se hace en esta publicación, hace uso de un modelo computacional anteriormente conocido en la programación funcional. Este modelo se basa en aplicar dos funciones básicas conocidas como Map (mapeo) y Reduce (reducci\'on). La función de mapeo consiste en aplicar una transformación o procedimiento a todos los datos, obteniendo así la entrada de la siguiente fase, reducción, la cual consiste en ''resumir", aplicando la misma función a diferentes partes de la salida de la fase de mapeo, obteniendo finalmente la salida deseada. Numerosas fases de mapeo y reducción pueden ser concatenadas en el orden que se quiera con el fin de producir el resultado esperado. Este modelo es el implementado en Hadoop pero con algunas peculiaridades\verapartado{desarrollo-hadoop-implementacion-modelo}.


\chapter{Desarrollo del proyecto}
	En este cap\'itulo se describir\'a el desarrollo del proyecto, el cual ha consistido en primer lugar en analizar las dos herramientas que se integraran, ECJ y Hadoop, con el objetivo de conocerlas en profundidad y as\'i poder integrarlas de la mejor manera posible. Una vez analizadas ambas, se explicar\'a el procedimiento llevado a cabo para su integración y finalmente los resultados obtenidos al ejecutar la evaluación de los individuos haciendo uso de Hadoop.
	\section{Estudio de ECJ como herramienta de c\'omputo evolutivo\label{desarrollo-ecj}}
		Las personas que se dedican a la investigación han apreciado desde siempre disponer de herramientas que faciliten la tarea de la investigación e implementen los algoritmos que necesitan utilizar. En este sentido surgen numerosas herramientas en el campo de la computación evolutiva, una bien conocida por la comunidad es ECJ. Esta herramienta ha sido llevada a cabo por el departamento de ciencias de la computación de la universidad de George Mason, US y ahora mismo se encuentra en su versi\'on 22.

Esta herramienta ha sido concebida para proporcionar una amplia flexibilidad que permita albergar numerosos tipos de problemas. Adem\'as, persiguiendo este mismo objetivo, ha sido desarrollada en el lenguaje de programación Java, lo que permite que pueda ser ejecutada en cualquier sistema operativo tanto Windows como Linux.

Algunas de sus características m\'as importantes y por las cuales ha adquirido la popularidad que posee son las siguientes:

\begin{itemize}
	\item Facilidad de analizar la ejecución con una \'util implementación de logging.
	\item Posibilidad de generar puntos de restauración, los cuales permiten detener la ejecución y reiniciarla en otro momento.
	\item Ficheros de parámetros anidados, lo cual permite jerarquizar la configuración y mantener m\'as clara su configuración.
	\item Ejecución multihilo de diferentes partes del proceso, esto da la posibilidad de paralizar el proceso en procesadores que posean varias unidades de procesamiento.
	\item Generación de n\'umeros pseudo-aleatorios de manera que se permite reproducir los resultados generados en anteriores ejecuciones.
	\item Soporte para diferentes técnicas evolutivas.
	\item Proceso de reproducción muy flexible representado por una jerarquía de operaciones.
\end{itemize}

Estas y otras características han hecho que ECJ se posicione como una de las herramientas favoritas para la investigación de la computación evolutiva. El hecho de su amplia utilización en la comunidad, ha motivado el uso de ECJ en este trabajo ya que de este modo el p\'ublico al que puede ir dirigido es m\'as amplio y m\'as grupos de investigación puedan beneficiarse de los resultados que de \'este se obtengan.

\figuraSinMarco{0.8}{imagenes/ecj-classes}{Clases que representan el proceso evolutivo  en ECJ}{ecj-classes}{}

La ejecución de cualquier algoritmo en ECJ esta guiada a través de los ficheros de configuración que deben ser anteriormente establecidos. Estos ficheros de configuración poseen una estructura jerárquica, esto permite que un fichero de configuración pueda incluir a otro/s y sobreescribir parámetros que hayan sido establecidos. Esta estructura permite que se pueda describir un problema con pocos parámetros ya que los que se utilicen de manera general estarán contenidos en otros que ser\'an simplemente incluidos. En estos ficheros de configuración se incluyen numerosos parámetros que describen el comportamiento del proceso evolutivo y existen algunos de ellos que son obligatorios especificar, como por ejemplo los que determinan que clase implementa cada una de las partes de la evolución.

\subsection{Proceso evolutivo}

En la documentación de ECJ se facilita un diagrama \ver{ecj-classes} que describe de forma clara como se implementa el proceso evolutivo en esta herramienta. Podemos observar que la clase que inicia el proceso evolutivo es Evolve. Esta clase genera un objeto que representa el estado de la evoluci\'on (EvolutionState), este objeto contiene cada una de las clases/etapas del proceso. La mayor parte de estas clases deben ser especificadas en los ficheros de configuración, de esta manera ECJ sabe que clase implementa cada una de las partes.

Como se ha comentado anteriormente \verapartado{analisis-evolutivos-paralelizacion}, la parte del proceso que suele ser m\'as costosa es la evaluación de individuos, representada \ver{ecj-classes} en ECJ por la clase Evaluator. En esta herramienta, \'esta es la clase encargada de la evaluación de cada uno de los individuos de la población, lo cual suele ser lo m\'as costoso computacionalmente y por consiguiente ser\'a la parte de ECJ donde se centre la implementación de la integración con Hadoop.
	\section{Estudio de Hadoop como herramienta de procesamiento masivo de informaci\'on\label{desarrollo-hadoop}}
		Apache Hadoop es un framework de software que soporta aplicaciones distribuidas con capacidades de procesamiento de cantidades masivas de información. Permite a las aplicaciones trabajar con miles de nodos y petabytes de datos. Hadoop se inspiró en los documentos de Google para MapReduce \cite{paper-mapreduce} y Google File System (GFS) \cite{paper-gfs}.

Hadoop es un proyecto de alto nivel Apache que está siendo construido y usado por una comunidad global de contribuyentes bastante amplia y activa, mediante el lenguaje de programación Java. Yahoo! ha sido el mayor contribuyente al proyecto, y usa Hadoop extensivamente en su negocio.

\figuraSinMarco{0.8}{imagenes/procesos-hadoop}{Procesos en los diferentes nodos de un cluster Hadoop}{procesos-hadoop}{}

Hadoop requiere tener instalados en los nodos del clúster la versi\'on 1.6 o superior del entorno de ejecuci\'on de Java (JRE) y SSH. Un clúster típico incluye un nodo maestro y múltiples nodos esclavos. El nodo maestro consiste en un proceso jobtracker (rastreador de trabajo), tasktracker (rastreador de tareas), namenode (nodo de nombres), y datanode (nodo de datos). Un esclavo o compute node (nodo de cómputo) consisten en un nodo de datos y un rastreador de tareas \ver{procesos-hadoop}. 

Como se puede apreciar, Hadoop puede ser claramente dividido en dos partes, el sistema de ficheros, HDFS, y la parte que implementa la parte de procesamiento de información, la capa MapReduce.

\subsection{El sistema de ficheros}

El Hadoop Distributed File System (HDFS) es un sistema de archivos distribuido, escalable y portátil escrito en Java para el framework Hadoop. Cada nodo en una instancia Hadoop típicamente tiene un único nodo de datos; un clúster de datos forma el clúster HDFS. La situación es típica porque cada nodo no requiere un nodo de datos para estar presente. 

Cada nodo sirve bloques de datos sobre la red usando un protocolo de bloqueo específico para HDFS. El sistema de archivos usa la capa TCP/IP para la comunicación; los clientes usan RPC para comunicarse entre ellos. El HDFS almacena archivos grandes, a través de múltiples máquinas. Consigue fiabilidad mediante replicado de datos a través de múltiples hosts, y no requiere almacenamiento RAID en ellos. Con el valor de replicación por defecto, 3, los datos se almacenan en 3 nodos: dos en el mismo rack, y otro en un rack distinto. Los nodos de datos pueden hablar entre ellos para reequilibrar datos, mover copias, y conservar alta la replicación de datos. HDFS no cumple totalmente con POSIX porque los requerimientos de un sistema de archivos POSIX difieren de los objetivos de una aplicación Hadoop, porque el objetivo no es tanto cumplir los estándares POSIX sino la máxima eficacia y rendimiento de datos. HDFS fue diseñado para gestionar archivos muy grandes. Algo que debe ser tenido en cuenta es que no proporciona alta disponibilidad, ya que la caída de su nodo maestro supone la caída del sistema de ficheros.

Su escalabilidad y su rendimiento se pueden llevar a cabo gracias a la falta de una de las características m\'as comunes en un sistema de ficheros, la capacidad de modificar el contenido de los ficheros que contiene, una característica no necesaria para el propósito para el que esta dise\~nado.

Aunque Hadoop tiene su propio sistema de ficheros, HDFS, no esta cerrado al uso de tan solo ese sistema, otros sistemas de ficheros son también compatibles como Amazon S3, CloudStore o FTP.

\subsection{La capa MapReduce}

Aparte del sistema de archivos, está el motor MapReduce, que consiste en un Job Tracker (rastreador de trabajos), para el cual las aplicaciones cliente envían trabajos MapReduce.

El rastreador de trabajos (Job Tracker) impulsa el trabajo fuera a los nodos Task Tracker disponibles en el clúster, intentando mantener el trabajo tan cerca de los datos como sea posible. Con un sistema de archivos consciente del rack en el que se encuentran los datos, el Job Tracker sabe qué nodo contiene la información, y cuáles otras máquinas están cerca. Si el trabajo no puede ser almacenado en el nodo actual donde residen los datos, se da la prioridad a los nodos del mismo rack. Esto reduce el tráfico de red en la red principal backbone. Si un Task Tracker (rastreador de tareas) falla o no llega a tiempo, la parte de trabajo se reprograma. El TaskTracker en cada nodo genera un proceso separado JVM para evitar que el propio TaskTracker mismo falle si el trabajo en cuestión tiene problemas. Se envía información desde el TaskTracker al JobTracker cada pocos minutos para comprobar su estado. El estado del Job Tracker y el TaskTracker y la información obtenida se pueden ver desde un navegador web proporcionado por Jetty.

\subsubsection{Un trabajo MapReduce}

\figuraSinMarco{1}{imagenes/mapreduce-job}{Fases de un trabajo MapReduce}{mapreduce-job}{}

Un trabajo MapReduce aplica dos funciones, como su nombre indica, una es una función Map (una operación a cada uno de los registros de entrada) y una función Reduce (una operación que resuma la salida de la función Map) para aplicar este modelo, Hadoop divide el proceso en diferentes fases, podemos resumirlas de la siguiente manera:

\begin{enumerate}
	\item Se divide los ficheros de entrada en tantas partes como n\'umero de tareas Map vayan a componer el trabajo.
	\item Cada tarea Map lee una de las partes y aplica a cada registro de entrada una función que define el usuario. La salida de esta fase son tuplas de clave-valor.
	\item La salida de cada tarea Map se organiza de forma local en el nodo que la ha ejecutado formando grupos, cada uno de ellos contiene todas las tuplas con la misma clave.
	\item Se transfieren todos los grupos con la misma clave al nodo que vaya a ejecutar la tarea de Reduce, de forma que las claves se entregan de manera ordenada.
	\item Se unen todos los grupos recibidos con la misma clave creando así la entrada de la tarea de Reduce.
	\item Se aplica la función de Reduce la cual recibe todas las tuplas con la misma clave y produce también tuplas de clave-valor.
	\item La salida de la operación de Reduce se almacena en el sistema de ficheros.
\end{enumerate}

Todo este proceso se ve resumido en el diagrama \ver{mapreduce-job} extraido de \cite{cloudera-mapreduce}, donde se muestran cada una de estas fases.

\label{fase-map-eval}
La idea de la integración es que la entrada del trabajo de MapReduce este compuesta por todos los individuos a evaluar, de manera que la función Map sea la que evalúe cada individuo, prescindiendo de la etapa de Reduce, de manera que se escriba en el sistema de ficheros directamente la salida de la fase de Map, el resultado de la evaluación de los individuos.












	\section{Diagramas de dise\~no}
		%\subsection{Diagrama de contexto}

\subsection{Diagrama de flujo de datos}

Con el fin de entender bien los flujos de datos que se producen en la implementaci\'on realizada se han elaborado dos diagramas de flujo que representan los dos enfoques seguidos en la integraci\'on de ECJ y Hadoop. El primero de ellos \ver{fases-evaluacion-un-trabajo} representa la implementación que consiste en un trabajo que eval\'ua toda la población, esta implementación se explica con mayor detalle m\'as adelante \verapartado{desarrollo-implementacion}. El segundo diagrama representa la implementación realizada para el problema de reconocimiento facial, la cual también se describe con detalle m\'as adelante \verapartado{problema-facerecognition-implementacion}.

\figuraSinMarco{0.9}{imagenes/fases-evaluacion-un-trabajo}{Flujo de información cuando se utiliza un trabajo para la evaluación}{fases-evaluacion-un-trabajo}{}

\figuraSinMarco{1.1}{imagenes/fases-evaluacion-trabajo-por-ind}{Flujo de información cuando se utiliza un trabajo para cada individuo}{fases-evaluacion-trabajo-por-ind}{}
	\section{Implementaci\'on}
		La herramienta de c\'omputo evolutivo ECJ es conocida entre otras cosas por su flexibilidad a la hora de poder desarrollar problemas en este ámbito, es por esto que la implementación que se plantea sigue la misma idea con el objetivo de que gran parte de los problemas que pueden ser implementados con esta herramienta, puedan hacer uso de la característica que se desarrolla en este trabajo de una forma sencilla.
\label{desarrollo-implementacion}

El propósito del trabajo es distribuir el costo computacional de la fase de evaluación de un algoritmo de computación evolutiva, es por esto que lo que debemos desarrollar es un Evaluator \ver{ecj-classes} el cual lleve a cabo la evaluación de los individuos.

Para poder hacer esto y que pueda ser este evaluador establecido en cualquier problema que se plantee en la herramienta, se debe crear una clase que extienda de la clase abstracta ec.Evaluator. Al heredar de una clase abstracta debemos implementar los métodos marcados como abstractos, los de la clase ec.Evaluator se muestran a continuación.

\begin{lstlisting}[language=Java]
    /** Evalua el fitness de una poblacion entera. */
    public abstract void evaluatePopulation(final EvolutionState state);

    /** Debe retornar true si la ejecucion del algoritmo debe detenerse por algun motivo.
    Un ejemplo es cuando se encuentra al individuo ideal */
    public abstract boolean runComplete(final EvolutionState state);
\end{lstlisting}

En ambos métodos recibimos un objeto EvaluationState el cual contiene el estado de la evaluación, eso engloba también a la población o subpoblaciones que es en lo que nosotros estamos interesados ya que debemos obtener cada individuo para evaluarlo.

Abordaremos en primer lugar el método runComplete por su simple implementación, la cual se puede observar a continuación:

\begin{lstlisting}[language=Java]
	public boolean runComplete(final EvolutionState state) {
		for (int x = 0; x < state.population.subpops.length; x++)
			for (int y = 0; y < state.population.subpops[x].individuals.length; y++)
				if (state.population.subpops[x].individuals[y].fitness.isIdealFitness())
					return true;
		
		return false;
	}
\end{lstlisting}

La implementación realizada se encarga únicamente de recorrer todos los individuos de la población (lo que conlleva recorrer cada subpoblaci\'on) y comprobar si el fitness de cada uno de los individuos es ideal haciendo uso del método .isIdealFitness(). En este caso, si alguno de los individuos posee un fitness ideal, se retorna true y si al recorrer todos los individuos ninguno es ideal, retornamos false.

\subsection{Evaluación de la población}

Antes de entrar un detalle de la implementación del método evaluatePopulation, debemos realizar algunas consideraciones. Para realizar la evaluación de los individuos, no es solo necesario las características que definen a cada individuo (genotipo), si no también aspectos como que función de evaluación utilizar o parámetros que determinan la forma de evaluar el individuo, estos y otros aspectos necesarios est\'an contenidos en el objeto que representa el estado de la evaluación (EvaluationState). Es por esto que en cada uno de los procesos que queramos evaluar los individuos, se debe tener acceso a este objeto, para tal propósito haremos uso de dos funciones clave que poseen ECJ y el sistema de ficheros de Hadoop (HDFS).

Para que todos los procesos tengan acceso al estado de la evaluación, haremos uso de la característica "puntos de restauración" que posee ECJ, esta característica permite serializar todo el estado del proceso evolutivo y enviarlo a través de la red o guardarlo en algún fichero, en principio esta característica se ideo con el fin de que se pueda detener el proceso y reanudarlo por donde iba cuando se desee pero en esta implementación se utilizar\'a para distribuir el estado del problema entre las diferentes m\'aquinas que llevar\'an a cabo la evaluación.

Por otro lado, utilizaremos una característica de HDFS denominada cache distribuida, esta característica permite almacenar ficheros de forma que se distribuyan por todos y cada uno de los nodos que conforman el cluster, lo cual es ideal en este caso ya que si almacenamos aquí el estado de la evolución, todos los procesos que se distribuyan podrán acceder a el.

Debemos tener en cuenta que el estado de la evaluación contiene todos los individuos de la población, lo cual no es necesario no es necesario distribuir ya que cada nodo evaluar\'a una parte de la población. Sabiendo \'esto, se distribuirá el estado de la evaluación sin la población, esta ser\'a proporcionada a través de los ficheros de entrada al problema, los cuales Hadoop se encarg\'a automáticamente de dividir y distribuir.

\subsubsection{Distribución del estado de la evaluación}

Una vez descrito el proceso necesario para la evaluación distribuida de los individuos, pasamos ahora a describir la implementación realizada del método evaluatePopulation, la cual establece varias etapas que describimos a continuación.

Con el objetivo de simplificar el desarrollo se ha implementado un cliente que lleva a cabo todas las tareas relacionadas con Hadoop:

\begin{lstlisting}[language=Java]
	HadoopClient hadoopClient = new HadoopClient(
											hdfs_address,
											hdfs_port,
											jobtracker_address,
											jobtracker_port);
			
	hadoopClient.setWorkFolder(work_folder);
\end{lstlisting}

El cliente implementado se ve caracterizado por cinco aspectos, las direcciones y puertos de los procesos principales de Hadoop y el directorio sobre el que se trabajar\'a en el sistema de ficheros (HDFS).

El primer paso que se lleva a cabo para la evaluación, es la creación y distribución del punto de restauración el cual almacena el estado de la evolución, esto se realiza en el siguiente fragmento de código:

\begin{lstlisting}[language=Java]
	//Extrae poblacion del estado de la evolucion
	LinkedList<Individual[]> individuals_tmp = new LinkedList<Individual[]>();
	for (Subpopulation subp : state.population.subpops){
		individuals_tmp.add(subp.individuals);
		subp.individuals = null;
	}
	
	//Crear punto de restauracion
	Checkpoint.setCheckpoint(state);
	
	//Restaurar poblacion en el estado de la evaluacion
	int i = 0;
	for (Individual[] individuals : individuals_tmp) {
		state.population.subpops[i++].individuals = individuals;
	}
			
	//Distribuir punto de restauracion
	hadoopClient.addCacheFile(new File("" + state.checkpointPrefix + "." + state.generation + ".gz"), true, true);
\end{lstlisting}

Como se mencionaba anteriormente, la población es eliminada del punto de restauración, para ello la extraemos del estado de la evaluación, creamos el punto de restauración y posteriormente la introducimos de nuevo. Por \'ultimo se distribuye el punto de restauración haciendo uso del cliente de Hadoop que utiliza la cache distribuida de HDFS para tal fin.

\subsubsection{Creación de la entrada del trabajo}

La población de individuos a evaluar conformar\'a la entrada del trabajo de MapReduce, por lo que debemos escribir cada uno de los individuos en HDFS para que después el trabajo puede leerlos desde ahi. Para ello, desde el evaluador llamamos al método createInput del cliente de Hadoop implementado. Este método consiste en lo siguiente:

\begin{lstlisting}[language=Java]
	Path input_file = new Path(work_folder.concat("/input/population.seq"));
	Writer writer = getSequenceFileWriter(input_file, IndividualIndexWritable.class, IndividualWritable.class);

	Subpopulation[] subpops = state.population.subpops;
	int len = subpops.length;
	for (int pop = 0; pop < len; pop++) {
		for (int indiv = 0; indiv < subpops[pop].individuals.length; indiv++) {
			if (!subpops[pop].individuals[x].evaluated){
				writer.append(new IndividualIndexWritable(pop, indiv), 
						      new IndividualWritable(state, subpops[pop].individuals[indiv]));
			}
		}
	}

	writer.close();
\end{lstlisting}

Creamos un fichero secuencial en el directorio de trabajo con el nombre "population.sql", el cual contendrá la población entera. Debemos tener en cuenta que la entrada de un trabajo de MapRedice est\'a compuesta por tuplas de clave-valor por lo que ese ser\'a el contenido de este fichero, donde la clave est\'a compuesta por dos números que identifican de forma inequívoca a cada individuo, lo población (pop) y el n\'umero de individuo dentro de ella (indiv), y el valor ser\'a el propio individuo que estará compuesto entre otras cosas por el genotipo.

Una vez creado el fichero se recorren todas las subpoblaciones e individuos y se a\~naden al fichero que contiene la población. Se debe comprobar si el individuo no est\'a evaluado, ya que pudiera provenir de una generación anterior y eso supone que ya est\'a evaluado y no es necesario volver a evaluarlo.

\subsubsection{Creación y ejecución del trabajo de MapReduce}

Una vez distribuido el estado de la evolución y generada la entrada del problema, podemos definir el trabajo de MapReduce para lanzarlo en Hadoop. Como se ha comentado anteriormente \verapartado{fase-map-eval}, se usar\'a la fase de Map y no la de reduce para la evaluación de individuos, para ello debemos definir la función que se llevar\'a a cabo en esta fase. Se dede implementar un mapper que ser\'a una clase que extienda de la clase org.apache.hadoop.mapreduce.Mapper y debe implementar el método map. En nuestro caso, el propósito es la evaluación de los individuos por lo que la definimos de la siguiente manera:

\begin{lstlisting}[language=Java]
	@Override
	protected void map(IndividualIndexWritable key, IndividualWritable value, Context context)
			throws IOException, InterruptedException {

		Individual ind = value.getIndividual();
		
		//Evaluar individuo
		SimpleProblemForm problem = ((SimpleProblemForm) state.evaluator.p_problem);
		problem.evaluate(state, ind, key.getSubpopulation(), 0);

		//Escribir fitness
		context.write(key, new FitnessWritable(state, ind.fitness));
	}
\end{lstlisting}

Como podemos observar, la clave y el valor que recibimos son los mismos que contienen nuestro fichero de entrada (IndividualIndexWritable y IndividualWritable), este método map ser\'a llamado por Hadoop con cada uno de los registros de los ficheros de entrada, en nuestro caso por cada individuo. Lo que hacemos es, en primer lugar, obtener el individuo desde el valor recibido, acceder al problema que estará definido por el usuario (problem) desde el estado de la evaluación (state) y una vez adquirido el problema, evaluamos al individuo. El resultado de la evaluación (el fitness) compondrá la salida de la función map junto al identificador del individuo, que si recordamos es la clave de entrada a la función map.

Definida la función, estamos en condiciones de crear el trabajo de MapReduce e iniciarlo. Mostramos en primer lugar el procedimiento a llevar a cabo y a continuación lo explicamos.

\begin{lstlisting}[language=Java]
	//Creacion del trabajo
	Job job = new Job(conf);
	job.setJarByClass(EvaluationMapper.class);
	
	//Configurar entrada
	job.setInputFormatClass(SequenceFileInputFormat.class);
	Path input_directory = new Path(work_folder.concat("/input"));
	SequenceFileInputFormat.addInputPath(job, input_directory);
	
	//Configurar fase Map
	job.setMapperClass(EvaluationMapper.class);
	job.setMapOutputKeyClass(IndividualIndexWritable.class);
	job.setMapOutputValueClass(FitnessWritable.class);

	//Configurar fase Reduce
	job.setNumReduceTasks(0);

	//Configurar salida
	job.setOutputFormatClass(SequenceFileOutputFormat.class);
	job.setOutputKeyClass(IndividualIndexWritable.class);
	job.setOutputValueClass(FitnessWritable.class);
	Path output_directory = new Path(work_folder.concat("/output"));
	hdfs.delete(output_directory, true);
	SequenceFileOutputFormat.setOutputPath(job, output_directory);
\end{lstlisting}

En primer lugar creamos el objeto que representa el trabajo de MapReduce, posteriormente configuramos la entrada indicando que es un fichero secuencial y el directorio de entrada. Ahora debemos definir las fases del trabajo, en primer lugar la fase Map, indicando la clase que implementa el método mostrado anteriormente (map) y en segundo lugar la fase de reduce de la cual vamos a prescindir por lo que tan solo debemos indicar que no habra ninguna tarea reduce. Por \'ultimo, configuramos la salida indicando de que tipos est\'a compuesta, los cuales coinciden con los tipos de salida del Mapper implementado, borramos el directorio de salida que contendría el resultado de la evaluación de la generación anterior y indicamos el directorio de salida.

Una vez configurado lo inicioamos de la siguiente manera:

\begin{lstlisting}[language=Java]
	job.waitForCompletion(true);
\end{lstlisting}

Esto enviar'a el trabajo al cluster que hayamos indicado en el cliente de Hadoop y evaluar\'a todos los individuos contenidos en los ficheros de entrada.

\subsubsection{Asignaci\'on de resultados}

El \'ultimo paso es asignarle a los individuos los resultados obtenidos de forma distribuida con el trabajo en MapReduce para continuar así de forma normal el proceso de la evoluci\'on. El trabajo ha generado en el directorio de salida un conjunto de ficheros (uno por cada proceso Mapper) que contienen la salida de la función map la cual est\'a compuesta por tuplas del identificador del usuario y el fitness calculado.

El procedimiento para recabar los resultados se lleva a cabo en el cliente de Hadoop implementado llamando al metodo readFitness y su implementación es la siguiente:

\begin{lstlisting}[language=Java]
	Path output_directory = new Path(work_folder.concat("/output"));

	// Obtenemos todos los ficheros que produjo el trabajo
	FileStatus[] output_files = hdfs.listStatus(output_directory);

	// Establecemos los fitness calculados
	IndividualIndexWritable key;
	SequenceFile.Reader reader;
	for (FileStatus output_file : output_files) {
		reader = new SequenceFile.Reader(conf, SequenceFile.Reader.file(output_file.getPath()));

		key = new IndividualIndexWritable();
		while (reader.next(key)) {
			Individual individual = state.population.subpops[key.getSubpopulation()].individuals[key.getIndividual()];
			
			//Establecemos el fitness
			reader.getCurrentValue(new FitnessWritable(state, individual.fitness));
			
			//Marcamos individuo como evaluado
			individual.evaluated = true;
		}

		reader.close();
	}
\end{lstlisting}

En primer lugar, obtenemos una lista con todos los ficheros de salida producidos y a continuación recorremos cada uno de ellos. Leemos cada registro del fichero y asignamos al individuo correspondiente (obtenemos subpoblacion y numero de individuo desde la clave leída) el fitness extraído y finalmente marcamos cada individuo como evaluado.
















	\section{Mejoras introducidas}
		//TODO
	\section{Despliegue de problemas utilizando la implementaci\'on}
		Describimos en esta sección cual es el procedimiento a seguir para desplegar en ECJ dos problemas bien conocidos en el mundo de la computación evolutiva. En primer lugar describimos como ejecutar el algoritmo genético MaxOne, y en segundo lugar el algoritmo de programación genética Parity. En ambos se hace inicialmente un despliegue que no hace uso de la integración con Hadoop y finalmente se explica como de una manera sencilla puede ser configurado para que la evaluación se haga de forma distribuida en un cluster Hadoop.
		\subsection{MaxOne}
			Para probar la implementación realizada, se ha configurado un problema sencillo que hace uso de la evaluación en un cluster Hadoop. El problema configurado se conoce con el nombre MaxOne y tiene como objetivo la construcción de una cadena de 1s, cada individuo esta representado por un conjunto de 1s y 0s teniendo todos los individuos una cadena de la misma longitud. La función de fitness es tan sencilla como contar el n\'umero de 1s en la cadena, el individuo ideal ser\'a aquel que su cadena contenga solo 1s.

Si quisiéramos ejecutar este problema con ECJ, sin hacer uso de la integración con Hadoop, debemos definir dos cosas. En primer lugar una clase que representa el problema, la cual en este caso tan solo implementa la función de evaluación, y en segundo lugar el fichero de configuración de ECJ.

A continuación mostramos una posible implementación de la clase que representa el problema:

\begin{lstlisting}[language=Java]
public class MaxOnes extends Problem implements SimpleProblemForm {
	public void evaluate(final EvolutionState state, final Individual ind, final int subpopulation, final int threadnum) {
		int sum = 0;
		
		BitVectorIndividual bv_ind = (BitVectorIndividual) ind;

		//Contamos numero de 1s
		for (int x = 0; x < bv_ind.genome.length; x++)
			sum += (bv_ind.genome[x] ? 1 : 0);
		
		//Establecemos el fitness
		((SimpleFitness) ind2.fitness).setFitness(state,
				(float) (((double) sum) / bv_ind.genome.length),
				sum == bv_ind.genome.length);  // es el individuo ideal?
		
		//Indicamos que ha sido evaluado
		bv_ind.evaluated = true;
	}
}
\end{lstlisting}

La imlementacion es tan b\'asica como convertir el individuo al tipo que le corresponde, BitVectorIndividual, contar el n\'umero de bits a true, establecer el fitness indicando si es el individuo ideal o no y por \'ultimo marcarlo como evaluado.

Una vez hecho esto debemos definir el fichero de configuración (con un nombre como config.params), el cual puede contener algo como lo siguiente:

\begin{lstlisting}[language=Java]
//Numero de hilos a usar para evaluacion y reproduccion
breedthreads	= 1
evalthreads	= 1

//Semilla de aleatoriedad
seed.0		= 4357

//Clases a utilizar para cada fase del proceso de evolucion
state		= ec.simple.SimpleEvolutionState
pop		= ec.Population
init		= ec.simple.SimpleInitializer
finish		= ec.simple.SimpleFinisher
breed	= ec.simple.SimpleBreeder
eval		= ec.simple.SimpleEvaluator
stat		= ec.simple.SimpleStatistics
exch		= ec.simple.SimpleExchanger

//Numero de generaciones maximo
generations		= 200
quit-on-run-complete	= true
checkpoint		= false

//Definimos una subpoblacion de BitVectorIndividuals de 200 bits y una reproduccion con seleccion por torneo de 2 individuos
pop.subpops		= 1
pop.subpop.0		= ec.Subpopulation
pop.subpop.0.size 		= 10
pop.subpop.0.duplicate-retries 	= 0
pop.subpop.0.species 		= ec.vector.BitVectorSpecies
pop.subpop.0.species.fitness 	= ec.simple.SimpleFitness
pop.subpop.0.species.ind	= ec.vector.BitVectorIndividual
pop.subpop.0.species.genome-size	= 200
pop.subpop.0.species.crossover-type	= one
pop.subpop.0.species.crossover-prob	= 1.0
pop.subpop.0.species.mutation-prob	= 0.01
pop.subpop.0.species.pipe			= ec.vector.breed.VectorMutationPipeline
pop.subpop.0.species.pipe.source.0		= ec.vector.breed.VectorCrossoverPipeline
pop.subpop.0.species.pipe.source.0.source.0	= ec.select.TournamentSelection
pop.subpop.0.species.pipe.source.0.source.1	= ec.select.TournamentSelection
select.tournament.size		= 2

//Indicamos la clase que define el problema (contiene la funcion de evaluacion)
eval.problem		= ec.app.tutorial1.MaxOnes
\end{lstlisting}

Una vez definidos estamos ficheros, podríamos ejecutarlo de forma local compilando el código fuente y ejecutando el siguiente comando:

\mostrarconsola{
	[usu@host src]\$ java ec.Evolve -file config.params
}

De este modo ejecutaríamos el un algoritmo evolutivo de forma local el cual terminar\'a su ejecución en cuanto encuentre a un individuo idea o alcance la generación 200.

Para este problema la ejecuci\'on es rápida, quizás un minuto o dos, ya que la función de evaluación es sencilla, y son pocos individuos, pero y si la evaluación fuera costosa o tuviéramos millones de individuos en la población? pues si disponemos de acceso a un cluster Hadoop podemos cambiar una linea y la evaluación se ejecutar\'a de forma distribuida haciendo uso de todos los recursos disponibles del cluster de manera que el tiempo de ejecución se reduzca notablemente. Esto podríamos conseguirlo cambiando el evaluador en el fichero de configuración, anteriormente lo establecimos a ec.simple.SimpleEvaluator (linea 14), si lo establecemos a ec.hadoop.HadoopEvaluator se ejecutar\'a haciendo uso de la implementación realizada, por lo que los individuos serán evaluados en todos los nodos del cluster Hadoop. Esta configuraci\'on ha sido probada obteniendo los mismos resultados que con la ejecución local.

Mas adelante \verapartado{resultados-maxone} se muestran los beneficios de utilizar esta solución, donde vemos que los tiempos se reducen considerablemente si lo comparamos con una ejecución secuencial.




















		\subsection{Parity}
			En esta sección vamos a configurar un sencillo problema de programación genética conocido por la comunidad por el nombre de Parity, el objetivo de este problema de programación genética es encontrar un programa que produzca un valor de la paridad par booleana dadas n entradas booleanas independientes. Por lo general, se utilizan 6 entradas booleanas (Parity-6), y el objetivo es que coincida con el valor del bit de paridad para cada una de las entradas de 2 elevado a 6 = 64 posibles.

Si queremos ejecutar este problema es ECJ debemos definir 3 cosas, implementar una clase que defina el problema y la cual pueda evaluar los individuos, definir otra clase que represente la paridad y ayuda a la transferencia de esta entre individuos de la población y por \'ultimo el fichero de par\'ametros de ECJ.

Comentamos en primer lugar la implementación de la clase que representa la paridad, utilizada en este caso para transferir un 0 o un 1 dependiendo de la paridad calculada.

\begin{lstlisting}[language=Java]
public class ParityData extends GPData {
	// Valor de retorno, debe ser siempre 1 o 0
	public int x;

	public void copyTo(final GPData gpd) {
		((ParityData) gpd).x = x;
	}
}
\end{lstlisting}

La clase que representa el problema, la cual lleva a cabo la evaluaci\'on de los individuos, se muestra a continuación.

\begin{lstlisting}[language=Java]
public class Parity extends GPProblem implements SimpleProblemForm {
	public int numBits;
	public int totalSize;
	public int bits; // data bits

	public void setup(final EvolutionState state, final Parameter base) {
		
		//Obtnemos el numero de bits a utilizar desde el fichero de parametros
		numBits = state.parameters.getIntWithMax(base.push(P_NUMBITS), null, 2, 31);

		//Calculamos la combinacion ma\'xima
		totalSize = 1;
		for (int x = 0; x < numBits; x++)
			totalSize *= 2;
	}

	public void evaluate(final EvolutionState state, final Individual ind,
			final int subpopulation, final int threadnum) {
		
		//Aqui se almacenara el valor de paridad retornado por el programa gen\'etico
		ParityData input = (ParityData) (this.input);

		int sum = 0;
		//Recorremos cada combinacion a probar
		for (bits = 0; bits < totalSize; bits++) {		
			//Comprobamos si es par o impar
			int tb = 0;
			for (int b = 0; b < numBits; b++)
				tb += (bits >>> b) & 1;
			tb &= 1; // now tb is 1 if we're odd, 0 if we're even

			//Ejecutamos el programa genetico que representa al individuo
			((GPIndividual) ind).trees[0].child.eval(state, threadnum, input, stack, ((GPIndividual) ind), this);

			//Si coinciden sumamos 1
			if ((input.x & 1) == tb)
				sum++;
		}

		//Establecemos el fitness en funcion del numero de aciertos
		KozaFitness f = ((KozaFitness) ind.fitness);
		f.setStandardizedFitness(state, (float) (totalSize - sum));
		f.hits = sum;
	}
}
\end{lstlisting}

En esta implementación, en primer lugar se ejecuta el método setup el cual obtiene el parámetro del numero de bits a utilizar en el problema y posteriormente se calcula el valor de la combinación maxima con ese numero de bits. Respecto a la evaluación de los individuos en el método evaluate, en primer lugar se creo un objeto que representa el valor devuelto por el programa genético, posteriormente se recorren todos los valores posibles y por cada uno se calcula la paridad del valor y se ejecuta el genetico, si ambos coinciden en el valor devuelto se incrementa el contador (sum). Finalmente se establece el fitness del individuo en funci\'on del n\'umero de aciertos.

Tras definir las dos clases anteriores, lo único que queda es el fichero de parámetros, a el cual podemos nombrar parity.params y cuyo contenido puede ser el siguiente:

\begin{lstlisting}[language=Java]
# Heredamos parametros basicos desde otro fichero
parent.0 = ../../gp/koza/koza.params

# Definimos todas las funciones a utilizar
gp.fs.size = 1
gp.fs.0.name = f0
gp.fs.0.func.0 = ec.app.parity.func.And
gp.fs.0.func.0.nc = nc2
gp.fs.0.func.1 = ec.app.parity.func.Or
gp.fs.0.func.1.nc = nc2
gp.fs.0.func.2 = ec.app.parity.func.Nand
gp.fs.0.func.2.nc = nc2
gp.fs.0.func.3 = ec.app.parity.func.Nor
gp.fs.0.func.3.nc = nc2
# Definimos tantas como queramos
...

# Definimos el problema
eval.problem = ec.app.parity.Parity
eval.problem.data = ec.app.parity.ParityData

# Numero de bits a utilizar
eval.problem.bits = 12
gp.fs.0.size = 16 # = eval.problem.bits + 4 para este problema
\end{lstlisting}

Con el objetivo de no repetir parámetros que suelen utilizarse con mucho frecuencia, heredamos desde un fichero de pararemos que contiene los m\'as usuales y sobreescribimos los que especifican el problema concreto, esto lo hacemos en la primera linea del fichero de parámetros con el parámetro parent y la ruta al fichero de parámetros que queremos heredar. A continuación se definen las funciones que pueden componer los programas genéticos producidos, indicamos el problema con las clases que hemos generado anteriormente y por ultimo definimos el numero de bits a utilizar en el problema.

Una vez definidos los ficheros necesarios, al igual que en el problema MaxOne podemos ejecutarlo si compilamos el código y ejecutamos el siguiente comando:

\mostrarconsola{
	[usu@host src]\$ java ec.Evolve -file parity.params
}

La configuración actual hace uso del evaluador definido en el fichero "../../gp/koza/ koza.params" del cual hereda y corresponde a ec.simple.SimpleEvaluator, este evaluador realizara una evaluación secuencial de los individuos pero si lo establecemos a ec.hadoop.HadoopEvaluator se ejecutar\'a en el cluster Hadoop. Para hacer esto podemos a\~nadir la siguiente linea al fichero de parámetros:

\begin{lstlisting}[language=Java]
eval = ec.hadoop.HadoopEvaluator
\end{lstlisting}

Esto sobreescribir\'a el valor del evaluador establecido en el fichero que heredamos y por consiguiente se hará uso del evaluador de Hadoop.

Mas adelante \verapartado{resultados-parity} se muestran los beneficios de utilizar esta solución, donde vemos que los tiempos se reducen considerablemente si lo comparamos con una ejecución secuencial.
	\section{Resultados}
		Analizamos ahora los resultados de la integración realizada centrándonos en uno de los problemas anteriores, Parity. Este problema engloba las circunstancias de ambos ya que el coste computacional de la evaluación en estos problemas es despreciable. Analizamos los tiempos obtenidos en la ejecución de este problema de programación genética, donde se evoluciona un programa hasta que se consigue generar una función de paridad que retorne el bit de paridad correcto en todas las combinaciones posibles.	
		\subsection{Millones de individuos: Parity}
			\label{resultados-parity}

Para los problemas donde el coste de la evaluación es despreciable, la paralelizaci\'on de esta fase cobra sentido cuando la cantidad de individuos a evaluar es importante. ECJ posee un evaluador que permite paralelizar la evaluación de los individuos de manera que se use los diferentes núcleos de procesamiento que posea la m\'aquina en la que se ejecute, de este modo podemos comparar los tiempos obtenidos con una ejecución secuencial y la multihilo. En la tabla que se muestra a continuación, mostramos tiempos en segundos que toma la etapa de evaluación en diferentes ejecuciones con diferente n\'umero de individuos y utilizando ejecuciones multihilo, estos tiempos corresponden a ejecuciones del problema Parity y la m\'aquina donde se ejecut\'o disponía de 8 núcleos de procesamiento.

\begin{table}[H]
  \begin{center}
    \begin{center}
    \begin{tabular}{l | c c c c}
    N\'umero de individuos & Secuencial & 2 hilos & 4 hilos & 8 hilos \\ \hline
    30.000 & 19 & 10 & 5 & 3\\
    50.000 & 31 & 16 & 8 & 4\\
    100.000 & 65 & 32 & 16 & 8\\
    300.000 & & & 47 & 25\\
    500.000 & & & 82 & 41\\
    1.000.000 & & & & 87\\
    1.500.000 & & & & 130\\
    \end{tabular}
    \end{center}
    \caption{Tiempos de ejecución secuencial y mulhilo}
    \label{tabla_tiempos_ecj}
  \end{center}
\end{table}

Observamos claramente como los tiempos son directamente proporcionales al n\'umero de individuos e indirectamente proporcionales al n\'uero de hilos que utilicemos para la evaluación. Como se puede apreciar con ejecuciones de millones de individuos, incluso utilizando 8 hilos de procesamiento, nos acercamos a tiempos del orden de minutos, teniendo en cuenta que esto debemos hacerlo por generación, la evaluación de los individuos empieza a ser un problema. 

\subsubsection{Resultados sin mejoras}

Abordamos ahora una ejecución utilizando la implementación realizada, de manera que los individuos se evalúen a lo largo de un cluster utilizando cada una de las m\'aquinas disponibles y los núcleos de procesamiento de cada una. Las ejecuciones realizadas hicieron uso de un cluster de 7 m\'aquinas interconectadas donde se encuentra desplegada la herramienta Hadoop. 

Como se mencion\'o anteriormente \verapartado{mejoras}, se hizo una implementación inicial y posteriormente se incluyeron mejoras. Los tiempos que se muestran \vertabla{tabla_tiempos_hadoop_sin_mejoras} son los obtenidos con la implementación inicial.
\label{resultados-mejoras}

\begin{table}[H]
  \begin{center}
    \begin{center}
    \begin{tabular}{l | c c c c}
    N\'umero de individuos & Tiempo de evaluaci\'on (segundos)\\ \hline
    30.000 & 25\\
    50.000 & 27\\
    100.000 & 32\\
    300.000 & 49\\
    500.000 & 58\\
    1.000.000 & 102\\
    1.500.000 & 198\\
    \end{tabular}
    \end{center}
    \caption{Tiempos de ejecución utilizando la integración con Hadoop sin los mejoras}
    \label{tabla_tiempos_hadoop_sin_mejoras}
  \end{center}
\end{table}

Se muestra una gráfica \ver{maxone-results-without-improvements} donde se observan los tiempos de la tabla anterior comparados con los tiempos de las diferentes ejecuciones sin la integración con Hadoop.

\figuraSinMarco{1}{imagenes/maxone-results-without-improvements}{Comparaci\'on de tiempos de evaluaci\'on sin y con la integraci\'on (sin mejoras)}{maxone-results-without-improvements}{}

Observamos que si lo comparamos con una ejecuci\'on secuencial (l\'inea roja), el uso de la implementaci\'on realizada mejora los tiempos con pocos miles de individuos, sin embargo, necesita alcanzar los 100.000 individuos para mejorar los tiempos de la ejecución con 2 hilos y hasta los 300.000 individuos para mejorar los de la ejecución con 4 hilos. Por otro lado observamos que los tiempos de la ejecución con 8 hilos nunca son alcanzados, llegando incluso a empeorar notablemente con una ejecución de un millón y medio de individuos.

 \subsubsection{Resultados con mejoras}

Analizamos ahora los tiempos tras la introducción de las mejoras descritas \verapartado{mejoras}, estos tiempos se muestran a continuación \vertabla{tabla_tiempos_hadoop_con_mejoras}.

\begin{table}[H]
  \begin{center}
    \begin{center}
    \begin{tabular}{l | c c c c}
    N\'umero de individuos & Tiempo de evaluaci\'on (segundos) \\ \hline
    30.000 & 19\\
    50.000 & 20\\
    100.000 & 24\\
    300.000 & 31\\
    1.000.000 & 65\\
    1.500.000 & 92\\
    2.000.000 & 108\\
    2.500.000 & 130\\
    3.000.000 & 141\\
    \end{tabular}
    \end{center}
    \caption{Tiempos de ejecución utilizando la integración con Hadoop con los mejoras}
    \label{tabla_tiempos_hadoop_con_mejoras}
  \end{center}
\end{table}

Al igual que lo hicimos con los resultados sin las mejoras, comparamos en una gráfica estos tiempos con la ejecución secuencial y con hilos \ver{maxone-results-with-improvements}.

\figuraSinMarco{1}{imagenes/maxone-results-with-improvements}{Comparaci\'on de tiempos de evaluaci\'on sin y con la integraci\'on (con mejoras)}{maxone-results-with-improvements}{}

Ahora observamos como los tiempos obtenidos ejecutando la fase de evaluación con Hadoop han mejorado notablemente, disminuyendo los tiempos de la ejecución secuencial, 2 y 4 hilos con un tamaño de la población no muy elevado, llegando a mejorar los tiempos de la ejecuci\'on con 8 hilos con una población no superior a los 500.000 individuos.



	
			
\chapter{En el mundo real: reconocimiento facial}		
	\label{problema-facerecognition}

El reconocimiento facial se utiliza en numerosas aplicaciones hoy en día tales como sistemas de seguridad, sistemas de identificación de personas, localización, etc. Las soluciones planteadas conllevan un coste computacional alto ya que el tratamiento de imágenes y la extracción de características son tareas costosas para un computador. En este cap\'itulo describimos la aplicación a este problema real y costoso la integración entre ECJ y Hadoop, esta integración reducir\'a los tiempos de procesamiento notablemente mediante la paralelizaci\'on utilizando Hadoop y proporcionar\'a un enfoque evolutivo con el cual mejorar la efectividad del algoritmo de reconocimiento facial.

Como se ha comentado en varias ocasiones en este documento, la integración de estas dos herramientas cobra sentido cuando el coste computacional es alto y esto se puede dar en dos situaciones, una en la que el tama\~no de la población sea elevado, lo cual no es usual en las poblaciones utilizadas con algoritmos evolutivos y otra en la que la evaluación de los individuos sea realmente costosa, este es el caso en el problema de reconocimiento facial.

\section{Descripci\'on del problema}

Surgen numerosas investigaciones que afrontan el problema de reconocimiento facial, un intento por resolver el problema se describe en \cite{paper-facerecognition} donde se presenta un sistema de clasificación de rostros haciendo uso de técnicas de clasificaci\'on no supervisadas, apoy\'andose en el an\'alisis de textura local con una t\'ecnica CBIR (Content Image Based Retrieval), por medio de la extracci\'on de la media, la desviaci\'on est\'andar y la homogeneidad sobre puntos de inter\'es de la imagen.

En este sistema se pueden diferenciar claramente dos fases, una es la fase de entrenamiento donde el sistema adquiere el conocimiento necesario para que posteriormente en la fase siguiente, recuperación, se pueda clasificar cada imagen (indicar a que persona pertenece) de forma correcta. El tiempo que toma ambas fases puede ser del orden de 4 minutos, estos tiempos son ya bastante reducidos gracias a que la implementaci\'on realizada hace uso de una librería de procesamiento de im\'agenes llamada OpenCV \cite{opencv}. La librería OpenCV consigue realizar procesamiento sobre matrices (las imágenes pueden ser tratadas como matrices) con una eficiencia notable.

\subsubsection{Base de datos de imágenes}

Antes de describir en detalle el proceso, explicamos en que consiste la base de datos de imágenes que utiliza este algoritmo como entrada. La base de datos est\'a formada por imágenes en condiciones ideales y cada una de ellas contiene información de diferentes puntos de interés situados en la imagen, todas tienen el mismo n\'umero de puntos de interés y corresponden cada uno de ellos a la misma parte del rostro. As\'i, por ejemplo, el punto n\'umero 67 de cada imagen corresponde a la punta de la nariz. Un ejemplo de localización de los puntos se puede observar en la imagen del rostro que contiene cada uno de los puntos de interés \ver{imagen-rostro}.

\subsection{Fase de entrenamiento}

En la primera fase llevada a cabo por el sistema de reconocimiento facial, se siguen diferentes pasos para obtener el conocimiento suficiente para poder posteriormente realizar la fase de consulta. Estos pasos se ven resumidos en el diagrama \ver{facerecognition-training} y son m\'as detalladamente explicados a continuación. 

\figuraSinMarco{1}{imagenes/facerecognition-training}{Arquitectura de la fase de entrenamiento}{facerecognition-training}{}

\figuraSinMarco{0.6}{imagenes/imagen-rostro}{Ejemplo de puntos de inter\'es sobre un rostro}{imagen-rostro}{}

\begin{enumerate}
	\item \textbf{Lectura de la imagen}. Se encarga de extraer las im\'agenes de la base de datos que ser\'an procesadas. La lectura de la imagen da como resultado una imagen en tres capas, correspondientes al espacio de color RGB.
	\item \textbf{Acondicionamiento.} Realiza el procesamiento previo de la imagen en cada una de las capas, ya que no se puede trabajar directamente con una imagen reci\'en adquirida, esta debe pasar por la etapa de acondicionamiento para que sea eliminada cualquier impureza generada, tanto por el dispositivo de adquisici\'on, la mala iluminaci\'on, el fondo, etc. Una vez eliminadas todas las impurezas, la imagen est\'a lista para su correcto an\'alisis.
	\item \textbf{Pasar al espacio de color HSI.} El espacio de color RGB no nos proporciona la informaci\'on necesaria o al menos no lo suficientemente clara para la extracci\'on de caracter\'isticas, por lo que la imagen RGB se debe pasar al espacio de color HSI. Este espacio de color nos proporcionar\'a la informaci\'on que necesitamos saber acerca de la imagen, como es la textura, luminosidad, saturaci\'on de color, etc. informaci\'on mas \'util que la que nos proporciona el espacio de color RGB, al final de este proceso tendremos nuevamente la imagen dividida en tres capas pero ahora en el espacio de color HSI.
	\item \textbf{Extracci\'on de caracter\'isticas.} Este bloque es el mas importante ya que es donde implementaremos la mayor parte del trabajo. Consta de tres etapas:
	\begin{enumerate}
		\item \textbf{Extracci\'on de los puntos de inter\'es.} Las bases de datos con las que se trabaja tienen los puntos de inter\'es marcados de manera manual, la localizaci\'on de los puntos de inter\'es est\'a contenida dentro de un documento de texto en el que cada l\'inea describe los puntos de inter\'es para cada una de las im\'agenes, por lo que nosotros debemos extraer los puntos de inte\'es con los que trabajar\'a nuestro sistema.
		\item \textbf{Extracci\'on de los puntos Surf (Speeded-Up Robust Features).} Este procedimiento viene implementado en la biblioteca de procesamiento digital de im\'agenes de OpenCV. Es un algoritmo de visi\'on por computador, capaz de obtener una representaci\'on visual de una imagen y extraer una informaci\'on detallada y espec\'ifica del contenido. Esta informaci\'on es tratada para realizar operaciones como por ejemplo la localizaci\'on y reconocimiento de determinados objetos, personas o caras, realizaci\'on de de escenas 3D, seguimiento de objetos y extracci\'on de puntos de inter\'es. Este algoritmo forma parte de la mencionada inteligencia artificial, capaz de entrenar un sistema para que interprete im\'agenes y determine el contenido. El Algoritmo SURF se present\'o por primera vez por Herbert Bay en ECCV 9"a conferencia internacional de visi\'on por computador celebrada en Austria en Mayo de 2006 \cite{bay2006surf}.
		\item \textbf{Extracci\'on de los rasgos estad\'isticos.} Los rasgos estad\'isticos son generados para cada uno de los puntos de inter\'es, de cada una de las capas del espacio de color HSI. A partir de un punto de inter\'es determinado en un pixel se genera una ventana p de tama\~no {Pi + p x Pi + p} p\'ixeles con el punto de inter\'es en el centro, dentro de esta ventana (vecindad) se extraer\'an tres valores de informaci\'on estad\'istica: media, desviaci\'on est\'andar y homogeneidad, por lo que al final obtendremos N puntos de inter\'es, 3 valores estad\'isticos para cada vecindad del punto de inter\'es y 3 capas por imagen, tendremos {N x 3 x 3} valores estad\'isticos de la imagen en cuesti\'on.
	\end{enumerate}
	\item \textbf{Conformar el vector patr\'on caracter\'istico de la imagen.} Los rasgos geom\'etricos SURF, se calculan de manera independiente, a los rasgos estad\'isticos CBIR.
		\begin{itemize}
			\item \textbf{Rasgos estad\'isticos CBIR.} Estos se calculan para cada uno de los puntos de inter\'es seleccionados de la base de datos, para cada una de las im\'agenes se seleccionan N puntos que se localicen dentro del rostro (ojos, nariz y boca), a cada uno de estos puntos se le extrae la informaci\'on estad\'istica (media, desviaci\'on est\'andar y homogeneidad), alrededor de una vecindad de tama\~no {Pi + p x Pi + p}
		\end{itemize}
		Al final de este proceso se obtendr\'a una matriz {Mj x 97}, con j = N\'umero de im\'agenes.
	\item \textbf{Normalizar los patrones.} A partir de los valores de la matriz M se toman los valores m\'aximos y m\'inimos de cada componente para normalizar los valores de la matriz entre 0 y 1, as\'i se obtendr\'a una nueva base de datos con la informaci\'on del contenido de las im\'agenes en valores normalizados.
	\item \textbf{Aplicar algoritmo k-Means con k = 10 experimental.} Se generan 10 clases (grupos), a partir de los valores de la base de datos normalizada, el valor de 10 es tomado como referencia ya que este valor es aplicado en la literatura y funciona muy bien, por lo que para comenzar con las pruebas tomaremos este valor de k = 10.
\end{enumerate}

\subsection{Fase de consulta}

Obtenidos los resultados de la fase de entrenamiento, la fase de consulta puede llevarse a cabo. Al igual que la anterior, \'esta se divide en diferentes pasos que se explican a continuación, estos se ven resumidos en la siguiente imagen \ver{facerecognition-query}.

\figuraSinMarco{1}{imagenes/facerecognition-query}{Arquitectura de la fase de consulta}{facerecognition-query}{}

Los pasos que se llevan acabo son similares a los de la fase de entrenamiento, exceptuando los tres \'ultimos: la normalización, la apliaci\'on del algoritmo k-NN y la obtenci\'on de las classes.

\begin{enumerate}
	\item \textbf{Lectura de la imagen.} Se encarga de extraer la imagen de la base de datos que ser\'a procesada. La lectura de la imagen da como resultado una imagen en tres capas, correspondientes al espacio de color RGB.
	\item \textbf{Acondicionamiento.} Realiza el procesamiento previo de la imagen en cada una de las capas. Una vez eliminadas todas las impurezas, la imagen est\'a lista para su correcto an\'alisis.
	\item \textbf{Pasar al espacio de color HSI.} El espacio de color RGB no nos proporciona la informaci\'on necesaria, por lo que la imagen RGB se debe pasar al espacio de color HSI.
	\item \textbf{Extracci\'on de caracter\'isticas.} Este bloque es el mas importante ya que es donde implementaremos la mayor parte del trabajo. Consta de tres etapas:
		\begin{enumerate}
		\item \textbf{Extracci\'on de los puntos de inter\'es.} Se consulta el fichero que contiene los puntos de inter\'es, leyendo la l\'inea que corresponde a la imagen a procesar, obteniendo de esta los puntos de inter\'es.
		\item \textbf{Extracci\'on de los puntos Surf (Speeded-Up Robust Features).} Es un algoritmo de visi\'on por computador, capaz de obtener una representaci\'on visual de una imagen y extraer una informaci\'on detallada y espec\'ifica del contenido. 
		\item \textbf{Extracci\'on de los rasgos estad\'isticos.} Los rasgos estad\'isticos son generados para cada uno de los puntos de inter\'es, de cada una de las capas del espacio de color HSI. A partir de un punto de inter\'es determinado se genera una ventana con el punto de inter\'es en el centro, dentro de esta ventana (vecindad) se extraer\'an tres valores de informaci\'on estad\'istica: media, desviaci\'on est\'andar y homogeneidad.
	\end{enumerate}
	\item \textbf{Conformar el vector patr\'on caracter\'istico de la imagen.} Los rasgos obtenidos de la imagen forman en esta etapa un vector que caracteriza la imagen.	
	\item \textbf{Normalizar los patrones.} A partir de los valores de la matriz M, se toman como referencia los valores m\'aximos y m\'inimos de cada componente (obtenidos en la fase de entrenamiento) para normalizar los valores del patr\'on entre 0 y 1, de la imagen en cuesti\'on.
	\item \textbf{Aplicar algoritmo k-NN con N = 5.} El valor del patr\'on es comparado contra los valores de las clases generadas en la etapa de entrenamiento, esta operaci\'on no es mas que una simple distancia euclidiana entre dos puntos. Al terminar las operaciones se obtienen las 5 im\'agenes mas parecidas a la imagen de consulta.
	\item \textbf{Asignaci\'on de classes.} En las 5 im\'agenes obtenidas se comprueba que la primera es ella misma (el patron debe ser el mismo) y que las 4 restantes corresponden a la misma persona. Con este metodo comprobamos la eficacia del algoritmo.
\end{enumerate}




















	\section{Modificación planteada: introducir evolución y paralelismo}
		Se proponen dos modificaciones al problema anterior: por un lado introducir evoluci\'on, para obtener los puntos de inter\'es de las im\'agenes que deben ser utilizados; por otro paralelizaci\'on, para reducir los tiempos de ejecuci\'on.

\subsubsection{Evoluci\'on: ECJ}

El sistema debe ser entrenado haciendo uso de una base de datos de imágenes en la que cada imagen est\'a caracterizada por unos puntos de interés. Todos estos puntos son analizados en la implementación descrita anteriormente. La modificación que se propone y es por esto que se plantea aquí, es la evolución de los puntos de interés a utilizar por el algoritmo, ya que puede ser que algunos de ellos caractericen las imagenes de mejor manera que otros o que algunos sirvan incluso para equivocar al sistema.

La configuraci\'on de este problema en ECJ guarda mucha relación con el problema de MaxOne \verapartado{desarrollo-maxone}, ya que los individuos también estar\'an representados por una cadena de 0s y 1s que indican en este caso si se utiliza o no el punto de interés, por lo que prácticamente lo único que cambia es la evaluación de individuos la cual es radicalmente m\'as compleja y costosa. De este modo, si nuestra base de datos asigna a cada imagen 10 puntos de inter\'es, el genotipo de nuestros individuos consistir\'a en una cadena de 0s y 1s de una longitud de 10 caracteres, indicando cada uno de ellos si el punto de inter\'es correspondiente a esa posición se utiliza (tendr\'a valor 1) o no (valor 0). Lo anteriormente explicado se expresa en un ejemplo en la imagen del rostro donde se seleccionan algunos puntos de interés \ver{facerecognition-face-evol}.

\figuraSinMarco{0.8}{imagenes/facerecognition-face-evol}{Ejemplo de individuo de la poblaci\'on}{facerecognition-face-evol}{}

Podemos analizar la modificaci\'on planteada m\'as en profundidad si lo describimos siguiendo los pasos de un algoritmo evolutivo. Estos pasos se enumeran a continuaci\'on y tambi\'en se muestran graficamente en la siguiente imagen \ver{facerecog-evo}.

\figuraSinMarco{1}{imagenes/evolucion}{Fases de la evoluci\'on en el problema de reconocimiento facial}{facerecog-evo}{}

\begin{enumerate}
	\item \textbf{Inicializaci\'on}: se genera una poblaci\'on inicial de tama\~no configurable con individuos generados de forma aleatoria. Cada uno es representado por una cadena de 0s y 1s de longitud el n\'umero de puntos de inter\'es que contengan las im\'agenes de la base de datos a utilizar.
	\item \textbf{Evaluaci\'on}: el c\'alculo de la aptitud de cada individuo consistirá en ejecutar el sistema de reconocimiento facial solo con los puntos que el genotipo del individuo indique que se deban utilizar, esto implica ejecutar tanto la fase de entrenamiento como la de consulta por cada individuo. La aptitud corresponderá al porcentaje de imágenes correctamente clasificadas en la fase de consulta. Todos estos trabajos se ejecutaran de forma paralela, esto se describe mejor en el siguiente punto donde se describe el modelo de paralelizaci\'on.
	\item Con cada \textbf{generaci\'on} se siguen los siguientes pasos:
	\begin{enumerate}
		\item \textbf{Selecci\'on}: se escogen dos individuos. Para obtener cada uno de ellos se hace una seleeci\'on por torneo donde se elige aleatoriamente dos individuos, se comparan sus aptitudes y se escoge el mejor de los dos.
		\item \textbf{Procreaci\'on}: la generaci\'on de nuevos individuos consiste en elegir un punto del genotipo de forma aleatoria que lo divida en dos partes e intercambiar estas partes entre los individuos elegidos. Esto genera dos nuevos individuos para la nueva poblaci\'on.
		\item \textbf{Mutaci\'on}: de los individuos generados, con una probabilidad peque\~na, se modifica un gen en su genoma. El gen a modificar se elige de forma aleatoria.
		\item \textbf{Evaluaci\'on}: ejecutar de nuevo con cada individuo el sistema de reconocimiento facial, ejecutando los diferentes trabajos de forma paralela.
	\end{enumerate}
\end{enumerate}

\subsubsection{Paralelismo: Hadoop}

\figuraSinMarco{1.1}{imagenes/fases-evaluacion-trabajo-por-ind}{Flujo de información cuando se utiliza un trabajo para cada individuo}{fases-evaluacion-trabajo-por-ind}{}

Si tenemos en cuenta el tiempo que se tarda en la ejecución secuencial del algoritmo (4 minutos) y realizamos una sencillas operaciones para una pequeña población de 10 individuos, cada generación tardar\'a en ser evaluada unos cuarenta minutos, si queremos evolucionarlo necesitaremos quizás decenas o centenas de generaciones lo cual conllevar\'ia un tiempo impracticable. Es por esto que adem\'as de la integración con ECJ, haremos uso de la integración con Hadoop para distribuir y paralizar el proceso y así poder hacer que la evaluación de cada generación sea cuestión de pocos minutos.

Como se ha comentado en la sección sobre la implementación de los problemas anteriores \verapartado{dos-modelos-para-dos-situaciones}, existe otro modelo que se puede llevar a cabo cuando la evaluación de cada uno de los individuos toma varios minutos y \'esta puede ser paralizada. \'Este es el caso del problema de reconocimiento facial que abordamos en este cap\'itulo en el que la evaluación de cada individuo puede tomar aproximadamente 4 minutos y consiste en aplicar un procesamiento a cada una de las imágenes de la base de datos de entrada, por lo que su paralelizaci\'on no se plantea muy complicada. Teniendo en cuenta esto, haremos uso de este modelo para la integraci\'on de este problema con Hadoop. La paralelizaci\'on se producirá a dos niveles ya que los trabajos se ejecutar\'an de manera simultánea y las tareas dentro de cada trabajo también. Anteriormente \ver{fases-evaluacion-trabajo-por-ind} se ha mostrado el flujo de información que se produce con esta implementación, ese diagrama puede ayudar a entender mejor las etapas que se llevan a cabo. 

Pasamos a describir cada una de las etapas que se llevan a cabo en este modelo para la evaluaci\'on una poblaci\'on.

\begin{itemize}
	\item \textbf{Se crea un hilo de ejecucui\'on por cada individuo}, cada uno de los hilos de forma paralela realiza los siguientes pasos:
	\begin{enumerate}
		\item \textbf{Lanzar trabajo para el entrenamiento}, en primer lugar se lanza un trabajo de Hadoop que lleva a cabo la fase de entrenamiento del problema de reconocimiento facial utilizando como entrada la base de datos de im\'agenes. Para ello utiliza la fase Map (tantas tareas como grupos de la base de datos de im\'agenes) del trabajo para:
		\begin{itemize}
			\item Acondicionamiento de la imagen.
			\item Pasar al espacio de color HSI.
			\item Extracci\'on de caracter\'isticas.
			\item Conformar el vector patr\'on caracter\'istico.
		\end{itemize}
		Y la de Reduce (solo una tarea) para:
		\begin{itemize}
			\item Normalizar los patrones.
			\item Aplicar algortimo k-Means
		\end{itemize}		
		\item \textbf{Lanzar trabajo para la fase de consulta}, con los resultados de la fase de entrenamiento y la base de datos de im\'agenes de puede ejecutar este trabajo en Hadoop. Para ello utiliza la fase Map (tantas tareas como grupos de la base de datos de im\'agenes) del trabajo para:
		\begin{itemize}
			\item Acondicionamiento de la imagen.
			\item Pasar al espacio de color HSI.
			\item Extracci\'on de caracter\'isticas.
			\item Conformar el vector patr\'on caracter\'istico.
			\item Normalizar el patr\'on.
		\end{itemize}
		Y la de Reduce (solo una tarea) para:
		\begin{itemize}
			\item Aplicar algortimo k-NN.
			\item Asignaci\'on de la clase (persona).
		\end{itemize}		
		\item \textbf{Asignaci\'on de resultados}, 
	\end{enumerate}
\end{itemize}

Estos pasos se explica con m\'as detalle en la secci\'on siguiente.

\subsection{Implementaci\'on} \label{problema-facerecognition-implementacion}

La entrada del trabajo de Hadoop estar\'a almacenada en el sistema de ficheros y estar\'a formada por la base de datos de imagenes. \'Esta ser\'a dividida y distribuida a lo largo del cluster para que las diferentes tareas que se ejecuten puedan acceder a ella.

Para hacer esto, el método encargado de evaluar los individuos en el Evaluator (evaluatePopulation) genera un hilo de ejecución de forma local para cada individuo de la poblaci\'on, de manera que cada uno de ellos se encarga de la evaluación de cada individuo. El propósito de estos hilos es lanzar los trabajos en Hadoop por lo que la mayoría del tiempo tan solo se dedican a esperar que la ejecución de los trabajos finalicen para proseguir con la ejecución normal del proceso evolutivo. La implementación de lo anteriormente explicado se puede observar en el siguiente fragmento de código:

\begin{lstlisting}[language=Java]
	Individual[] inds = subpops[0].individuals;

	//Creamos un thread por cada individuo
	EvaluateIndividual[] threads = new EvaluateIndividual[inds.length];
	for (int ind = 0; ind < inds.length; ind++)
		threads[ind] = new EvaluateIndividual(state, 
								conf, 
								state.generation, 
								ind, 
								(BitVectorIndividual) inds[ind]);
			
	//Iniciamos los threads
	for (int ind = 0; ind < inds.length; ind++)
		threads[ind].start();
			
	//Esperamos a la finilizacion de todos
	for (int ind = 0; ind < inds.length; ind++)
		threads[ind].join();
\end{lstlisting}

Pasamos ahora a describir que es lo que hace cada uno de los hilos. Al ser hilos, deben implementar un m\'etodo llamado run(), este método se encarga de lanzar en primer lugar el trabajo que realiza la fase de entrenamiento del sistema de reconocimiento facial y una vez que acaba correctamente este trabajo, lanza otro encargado de la fase de consulta, estos trabajos no pueden ser paralizados ya que la fase de consulta requiere de los resultados de la fase de entrenamiento. La implementación se puede observar a continuación:

 \begin{lstlisting}[language=Java]
 	//Lanzamos trabajo de entrenamiento, en caso de no terminar exitosamente lanzamos una excepcion
	if(traningJob != null && !traningJob.run())
		throw new RuntimeException("Individual " + ind + ": there was a problem during the training phase");
		
	//Lanzamos trabajo de consulta y obtenemos el fitness
	Float fitness = queryJob.run();
	if(fitness == null)
		throw new RuntimeException("Individual " + ind + ": there was a problem during the query phase");
			
	//Asignamos el fitness calculado y marcamos el individuo como evaluado
	((SimpleFitness) individual.fitness).setFitness(state, fitness, fitness >= 1F);
	individual.evaluated = true;	
\end{lstlisting}

\subsubsection{Trabajo de la fase de entrenamiento}

Este trabajo implementa las dos fases de un trabajo de Hadoop, la fase de Map y la de Reduce por lo que describiremos en qu\'e consiste cada una. La primera fase, la de Map, tiene como implementación el siguiente fragmento de c\'odigo:

 \begin{lstlisting}[language=Java]
	@Override
	protected void map(NullWritable key, ImageWritable image, Context context)
			throws IOException, InterruptedException {
		
		MatE parameters = image.getParameters(windows_size);
		
		context.write(NullWritable.get(), new MatEWithIDWritable(image.getId(), parameters));
	}
\end{lstlisting}

Como se puede observar, lo que recibe nuestra tarea map son tuplas de NullWritable y ImageWritable, en este caso la clave no la utilizamos por que la establecemos a NullWritable y lo que recibimos en el valor es una imagen a procesar. De lo que se encarga esta función es de extraer los parámetros de cada uno de los puntos de interés (solo los que el genotipo del individuo indique a 1) y almacenarlos en una matriz de OpenCV (MatE), la cual junto con el identificador de la imagen, formaran la salida de la función map. El c\'alculo de estos parámetros conlleva numerosas operaciones con cada imagen las cuales no se describen ya que se considera no es el propósito de este trabajo, sin embargo estas pueden ser consultados en el código fuente que se proporciona.

Una vez finalizada la fase de Map tenemos todos los parámetros de cada imagen, de modo que la fase de Reduce puede comenzar, en nuestro caso la fase de Reduce se lleva a cabo en un solo nodo (no en uno en concreto, si no en alguno de los que componen el cluster) ya que este necesita toda la información producida por la fase de Map para obtener sus resultados. Procedemos ahora a mostrar la implementación de la fase de Reduce.

 \begin{lstlisting}[language=Java]
@Override
	protected void reduce(
			NullWritable key,
			Iterable<MatEWithIDWritable> values,
			Context context)
			throws IOException, InterruptedException {

		//Unimos todos los parametros recibidos por orden en una sola matriz
		MatE matRef = new MatE();
		List<MatEWithIDWritable> mats = new LinkedList<MatEWithIDWritable>();
		for (MatEWithIDWritable mat : values) {
			MatEWithIDWritable tmp = new MatEWithIDWritable();
			mat.copyTo(tmp);
			mat.release();
			mats.add(tmp);
			
			number_of_images++;
		}
		Collections.sort(mats);
		Core.vconcat((List<Mat>)(List<?>) mats, matRef);

		//Obtenemos valores maximos por columnas y la normalizamos
		MatE max_per_column = matRef.getMaxPerColumn();
		matRef = matRef.normalize(max_per_column);
		
		//Calculamos Kmeans
		MatE centers = new MatE();
		MatE labels = new MatE();
		TermCriteria criteria = new TermCriteria(TermCriteria.EPS + TermCriteria.MAX_ITER, 10000, 0.0001);
		Core.kmeans(matRef, num_centers , labels, criteria, 1, Core.KMEANS_RANDOM_CENTERS, centers);
		
		//Generamos matriz de indices de texturas
		MatE textureIndexMatriz = new MatE(Mat.zeros(number_of_images, num_centers, CvType.CV_32F));
		int pos;
		for (int i = 0; i < number_of_images; i++) {
			pos = number_of_poi * i;
			for (int j = pos; j < pos + number_of_poi; j++) {
				double[] valor = textureIndexMatriz.get(i, (int) labels.get(j, 0)[0]);
				valor[0] = valor[0] + 1;
				textureIndexMatriz.put(i, (int)labels.get(j,0)[0], valor);
			}
		}
		
		//Producimos la salida
		context.write(NullWritable.get(), new TrainingResultsWritable(max_per_column, centers, textureIndexMatriz));
	}
\end{lstlisting}

Podemos observar como el método reduce recibe como entrada una lista (values) la cual contiene todas matrices producidas por la fase de Map. Lo que hacemos en el reduce es unir todas esas matrices de parámetros en una sola, calcular los máximos por columna, normalizarla, calcular Kmeans y generar una matriz de \'indices de textura. Finalmente escribimos todos los resultados los cuales necesitaremos para el proximo trabajo, el de consulta.

\subsubsection{Trabajo de la fase de consulta}

De la etapa de consulta del sistema de reconocimiento facial es de lo que se encarga el trabajo de Hadoop que ahora vamos ha describir. Al igual que el anterior, este utiliza la fase de Map y la de Reduce (una sola tarea de reduce) para obtener finalmente el fitness del individuo. Mostramos en primer lugar la implementación de la fase de Map.

 \begin{lstlisting}[language=Java]
 	private TrainingResultsWritable trainingResults;
 
	@Override
	protected void setup(Context context) throws IOException, InterruptedException {

		//Obtenemos los resultados del entrenamiento
		String file = context.getConfiguration().get(EvaluateIndividual.INDIVIDUAL_DIR_PARAM).concat("training/part-r-00000");
		Reader reader = new Reader(fs.getConf(), Reader.file(file));
		reader.next(key, trainingResults);
	}
	
	@Override
	protected void map(NullWritable key, ImageWritable image, Context context)
			throws IOException, InterruptedException {;
		
		//Normalizamos los parametros de la imagen
		MatE normalized_params = image.getParameters(windows_size).normalize(trainingResults.getMaxPerCol());
		
		//Obtenemos centroides con Knn
		MatE idCenters = knn(trainingResults.getCenters(), normalized_params);

		//Obtenemos vector de consulta		
		MatE queryVector = MatE.zeros(1, trainingResults.getCenters().rows(), CvType.CV_64F);
		for (int poi_index = 0; poi_index < idCenters.rows(); poi_index++) {
			int x = (int) idCenters.get(poi_index, 0)[0];
			double y = queryVector.get(0, x)[0];
			queryVector.put(0, x, y + 1);
		}
		
		context.write(new MatEWritable(trainingResults.getTextureIndexMatrix()), 
						new MatEWithIDWritable(image.getId(), queryVector));
	}
\end{lstlisting}

Al igual que el anterior trabajo, el de entrenamiento, la fase de Map recibe las imágenes como entrada, pero este difiere del anterior en que implementa el método setup() el cual se ejecuta antes del iniciar la ejecución de la tarea y se encarga de obtener los resultados producidos por el trabajo anterior los cuales están en el directorio del individuo. Respecto al método map, en primer lugar obtiene los parámetros de la imagen recibida y los normaliza con respecto los máximos obtenidos del trabajo anterior, una vez normalizados, obtiene los centroides haciendo uso de un algoritmo Knn y por \'ultimo genera un vector de consulta que junto a la matriz de \'indices de textura del trabajo anterior compondrán la salida de la fase de Map.

Abordamos ahora la fase de Reduce del trabajo, cuya implementación se muestra a continuación.

 \begin{lstlisting}[language=Java]
	//Relacion de imagen-clase (persona)
	HashMap<Integer, Integer> img_class;
	
	@Override
	protected void setup(Context context) throws IOException, InterruptedException {
		img_class = getImagesClass(conf);
	}
	
	@Override
	protected void reduce(MatEWritable textureIndexMatrix, Iterable<MatEWithIDWritable> queryVectors, Context context)
			throws IOException, InterruptedException {

		//Unimos todos los vectores de consulta
		MatE query_mat = new MatE();
		List<MatEWithIDWritable> vectors = new LinkedList<MatEWithIDWritable>();
		for (MatEWithIDWritable queryVector : queryVectors) {
			MatEWithIDWritable tmp = new MatEWithIDWritable();
			queryVector.copyTo(tmp);
			queryVector.release();
			
			vectors.add(tmp);
		}
		Collections.sort(vectors);
		Core.vconcat((List<Mat>)(List<?>) vectors, query_mat);
		
		//Obtenmos los ids mas cercanos
		MatE nearestIds = knn(textureIndexMatrix, query_mat, num_nearest);
		
		//Calculamos matriz de confusion		
		MatE confusionMatrix = generateConfusionMatrix(nearestIds, num_nearest);

		//Obtenemos porcentaje de acierto
		float suma=0;
		for (int i=0;i<confusionMatrix.rows();i++)
			suma = suma + (float)confusionMatrix.get(i,i)[0];
		
		float percentage = suma / (float)vectors.size() / (float)(num_nearest - 1);
		
		context.write(NullWritable.get(), new FloatWritable(percentage));
	}
\end{lstlisting}

En esta fase, al igual que la anterior, implementamos el método setup() encargado en este caso de obtener la clase (persona) que corresponde a cada una de las imágenes. Respecto al método reduce, une en primer lugar por orden todos los vectores de consulta recibidos, obtiene los ids de las imagenes que m\'as se le asemejan, calcula la matriz de confusi\'on y por \'ultimo obtiene el porcentaje de aciertos. Este porcentaje al fitness del individuo y por consiguiente ser\'a la salida del trabajo de consulta.














	\section{Resultados}
		\label{resultados-facerecognition}

Analizamos los resultados obtenidos al ejecutar un problema en el que la evaluación de cada individuo toma varios minutos. En una ejecuci\'on secuencial, los tiempos de la evaluación de cada generaraci\'on puede llevar horas dependiendo del tamaño de la población.

La implementaci\'on secuencial del algoritmo de reconocimiento facial utiliza la librer\'ia OpenCV para el tratamiento de las im\'agenes. El uso de esta librer\'ia ayuda a reducir notablemente el tiempo de procesamiento, as\'i la fase de entrenamiento en la ejecuci\'on secuencial necesita sobre unos 127 segundos y la fase de consulta unos 140 segundos. 

En primer lugar se realiz\'o la integraci\'on del problema de reconocimiento facial con Hadoop para paralelizar el procesamiento de las im\'agenes haciendo uso de esta herramienta. El nivel de paralelizaci\'on depende de como se divida la entrada del problema ya que se ejecutar\'an tantas tareas en paralelo como divisiones de la entrada existan \ver{fases-evaluacion-trabajo-por-ind}. Esta idea nos podr\'ia llevar al convencimiento de que obtendremos mejores tiempos mientras m\'as tareas tengamos, pero la creaci\'on de la tarea, distribuci\'on de la entrada al trabajo y la posterior destrucci\'on de la tarea conllevan tiempos que deben ser considerados. Es por esto que se han realizado ejecuciones para encontrar el mejor n\'umero de tareas que se ejecuten de forma paralela. Los tiempos obtenidos en segundos se muestran en una tabla \vertabla{tabla-tiempos-num-grupos}.

\begin{table}[H]
  \begin{center}
    \begin{center}
    \begin{tabular}{l | c c c c c c}
    Grupos & 10 & 15 & 20 & 25 & 30 & 40 \\ \hline
    Entrenamiento (s) & 63 & 42 & 39 & 41 & 45 & 30\\
    Consulta (s) & 57 & 46 & 43 & 46 & 50 & 54\\
    \end{tabular}
    \end{center}
    \caption{Tiempos de ejecución en Hadoop en funci\'on del n\'umero de grupos}
    \label{tabla-tiempos-num-grupos}
  \end{center}
\end{table}

Se graficamos estos resultados \ver{grafica-num-grupos} compar\'andolos con los tiempos de la ejecuci\'on secuencial podemos observar facilmente dos cosas. En primer lugar, el hecho de que el n\'umero de grupos mas apropiado en el que se debe dividir la entrada son 20. Por otro lado vemos como la ganancia de tiempo con respecto a la ejecuci\'on secuencial es significativa. Si sumamos los tiempos de ambas fases en ambas implementaciones (tiempos con 20 grupos) y los comparamos, vemos como la integraci\'on ha conseguido que se ejecute el algoritmo en un 30\% del tiempo de la ejecuci\'on secuencial.

\figuraSinMarco{0.8}{imagenes/facerecognition-results-num-splits}{Comparaci\'on de tiempos en funci\'on del numero de grupos de la entrada}{grafica-num-grupos}{}

Tras la integraci\'on con Hadoop, se procedi\'o con la adaptaci\'on del algoritmo de reconocimiento facial a un algoritmo evolutivo utilizando ECJ. Esta integraci\'on provoca que cada individuo de la poblaci\'on ejecute dos trabajos, el de entrenamiento y consulta, y todos ellos de forma paralela por lo que se produce otro nivel de paralelizaci\'on. 

Con el fin de probar la escalabilidad del sistema se ha lanzado ejecuciones con diferentes n\'umero de nodos y el mismo n\'umero de individuos (10) registrando los tiempos obtenidos \vertabla{tabla-dif-num-nodos}.

\begin{table}[H]
  \begin{center}
    \begin{center}
    \begin{tabular}{l | c c c}
    N\'umero de nodos & 2 & 4 & 6 \\ \hline
    Evaluaci\'on (min) & 25 & 11.5 & 6.4\\
    \end{tabular}
    \end{center}
    \caption{Tiempos de evaluaci\'on de la poblaci\'on en Hadoop con diferente n\'umero de nodos para 10 individuos}
    \label{tabla-dif-num-nodos}
  \end{center}
\end{table}

\figuraSinMarco{0.6}{imagenes/facerecognition-results-num-nodes}{Comparaci\'on de tiempos en funci\'on del n\'umero de nodos}{grafica-num-nodos}{}

Graficados estos datos \ver{grafica-num-nodos} observamos como la escalabilidad del sistema mejora incluso los tiempos ideales, esto nos indica que la inclusi\'on de nuevos nodos en el cluster en el que lo corramos (f\'acil tarea) provocar\'a que los tiempos se reducir\'an acorde al n\'umero de nodos que a\~nadamos. Para observar mejor la escalabilidad del sistema se ha a\~nadido una l\'inea de tiempo ideal correspondiente a la mitad del tiempo de 2 nodos para 4 nodos y a la tercera parte del tiempo de 2 nodos para 6 nodos. Adem\'as, se ha a\~nadido al gr\'afico el tiempo que tomar\'ia esta ejecuci\'on de forma secuencial (evaluar un individuo {= 127 + 140 = 267 s}, evaluar 10 {= 10 x 267 / 60 = 44.5 min}) para observar lo que supone el uso de Hadoop.

El hecho de que se mejoren los tiempos ideales es de extra\~nar, pero debemos tener en cuenta que Hadoop hace uso de numerosos servicios que se pueden ver sobrecargados si todo se ejecuta en un n\'umero reducido de nodos. Por esta raz\'on es f\'acil que se produzcan este tipo de anomal\'ias en los tiempos analizados.

Analizamos ahora ejecuciones con diferentes tama\~nos de poblaci\'on y el mismo n\'umero de nodos (6). Se espera que los tiempos de evaluaci\'on sean directamente proporcionales al tama\~no de la poblaci\'on.

\begin{table}[H]
  \begin{center}
    \begin{center}
    \begin{tabular}{l | c c c c}
    N\'umero de individuos & 10 & 20 & 30 \\ \hline
    Evaluaci\'on (min) & 6.4 & 14.5 & 22.1 \\
    \end{tabular}
    \end{center}
    \caption{Tiempos de evaluaci\'on de la poblaci\'on en Hadoop con diferente n\'umero de individuos en 6 nodos}
    \label{tabla-dif-num-indiv}
  \end{center}
\end{table}

\figuraSinMarco{0.6}{imagenes/facerecognition-results-num-indiv}{Comparaci\'on de tiempos en funci\'on del n\'umero de individuos}{grafica-num-indiv}{}

Al igual que hemos hecho con los resultados anteriores, lo comparamos con los tiempos de ejecuciones secuenciales y tiempos ideales \ver{grafica-num-indiv}. Vemos como los beneficios del uso de la integraci\'on con Hadoop son significativos, adem\'as se obtienen resultados similares a los ideales.











\chapter{Manual de usuario}
	\section{Obtenci\'on}
		El primer paso para la utilización de la solución implementada, no puede ser otro que obtenerla. Con el fin de que la comunidad pueda conseguirla sin problema, se ha creado un repositorio p\'ublico desde donde se puede descargar.

El repositorio est\'a localizado en la conocida p\'agina por los desarrolladores GitHub (\url{https://github.com}), en ella se encuentran numerosos proyectos de código abierto. Todos los proyectos almacenados en esta p\'agina hacen uso de una herramienta de control de versionado llamada Git, esta herramienta ha sido utilizada en este proyecto y su utilizaci\'on se explica m\'as adelante.

\insertaraclaracion{?`Qu\'e es Git?}{
	\qquad Es un software de control de versiones diseñado por Linus Torvalds, pensando en la eficiencia y la confiabilidad del mantenimiento de versiones de aplicaciones cuando estas tienen un gran número de archivos de código fuente. Hay algunos proyectos de mucha relevancia que ya usan Git, en particular, el grupo de programación del núcleo Linux.
	\qquad Algunas de sus características m\'as destacadas son:
	\begin{itemize}
		\item Fuerte apoyo al desarrollo no lineal, por ende rapidez en la gestión de ramas y mezclado de diferentes versiones.
		\item Gestión distribuida. Git le da a cada programador una copia local del historial del desarrollo entero, y los cambios se propagan entre los repositorios locales. 
		\item Los almacenes de información pueden publicarse por HTTP, FTP, rsync o mediante un protocolo nativo, ya sea a través de una conexión TCP/IP simple o a través de cifrado SSH.				\item Los repositorios Subversion y svk se pueden usar directamente con git-svn.
		\item Gestión eficiente de proyectos grandes.
	\end{itemize}
}

El proyecto se puede encontrar a través del buscador del sitio escribiendo ''ecj\_hadoop", o accediendo directamente al repositorio siguiendo esta dirección: \url{https://github.com/dlanza1/ecj_hadoop} . Una vez en el repositorio, el proyecto se puede obtener de dos formas, una es pulsando en el botón "Download ZIP" (situado en la columna de la derecha), y una vez descargado descomprimir el fichero, y la otra forma es clonar el repositorio con Git (crear un repositorio igual de forma local).

Para clonar el repositorio con Git se debe tener instalada esta herramienta en el sistema \cite{instalacion_git} o utilizar un entorno de desarrollo como Eclipse el cual la trae incluida. 

Para clonarlo desde un entorno de desarrollo como Eclipse \cite{eclipse} el procedimiento suele ser el de importar un proyecto pero con la diferencia de que se debe indicar que la fuente es un repositorio Git, nos solicitar\'a el repositorio que queremos clonar y es ah\'i donde debemos escribir la URL del repositorio. Algo que debe ser aclarado es que esta operación crear\'a un repositorio local, no genera el proyecto en el entorno de desarrollo, lo cual se explica en la sección siguiente.

Si de otro modo, tenemos la herramienta instalada en el sistema, debemos en primer lugar abrir una consola, posteriormente dirigirnos al directorio donde queremos crear el repositorio local y por \'ultimo clonar el repositorio con el siguiente comando:

\mostrarconsola{
	[usu@host repo]\$ git clone https://github.com/dlanza1/ecj\_hadoop
}

Una vez clonado tendremos una copia idéntica del repositorio la cual contiene la \'ultima versi\'on del proyecto adem\'as de toda la información necesaria para que funcione el sistema de versionado utilizada por Git. 






	\section{Importar a entorno de desarrollo}
		Esta sección tiene como objetivo explicar el procedimiento para importar el proyecto en un entorno de desarrollo, esto no es necesario para su ejecución por lo que no estén interesados en modificar/ampliar el proyecto pueden continuar la lectura en la sección siguiente (Compilación).

En el apartado anterior se ha explicado como obtener el proyecto, pero lo obtenido es básicamente los ficheros de código fuente, no se incluyen proyectos de entornos de desarrollo ni ejecutables.

Si acabamos de descargar/clonar el proyecto, no lo tendremos importado en nuestro entorno, para llevar a cabo esta operación tan solo debemos dirigirnos al menu de importación de proyectos, solucionar que es un proyecto Maven y seleccionar el directorio que contiene el proyecto. 

Otra opción, teniendo la herramienta Maven instalada en el sistema \cite{instalacion_maven}, es generar primero el proyecto del entorno de desarrollo y posteriormente importarlo como un proyecto normal. Para generar el proyecto del entorno de desarrollo debemos abrir una consola, dirigirnos al directorio donde se sitúa el proyecto y ejecutar el comando Maven correspondiente a nuestro entorno, por ejemplo para Eclipse debemos ejecutar el siguiente comando:

\mostrarconsola{
	[usu@host repo]\$ mvn eclipse:eclipse
}

Esto nos generar\'a el proyecto Eclipse y podremos importarlo como cualquier otro proyecto.
	\section{Compilaci\'on}
		En esta sección se describe como 

\insertaraclaracion{?`Qu\'e es Maven?}{
	\qquad Maven es una herramienta de software para la gestión y construcción de proyectos Java creada por Jason van Zyl, de Sonatype, en 2002. Tiene un modelo de configuración de construcción simple, basado en un formato XML. Es un proyecto de nivel superior de la Apache Software Foundation.
	
	\qquad Maven utiliza un Project Object Model (POM) para describir el proyecto de software a construir, sus dependencias de otros módulos y componentes externos, y el orden de construcción de los elementos. Viene con objetivos predefinidos para realizar ciertas tareas claramente definidas, como la compilación del código y su empaquetado.
	
	\qquad El motor incluido en su núcleo puede dinámicamente descargar plugins de un repositorio, el mismo repositorio que provee acceso a muchas versiones de diferentes proyectos. Este repositorio pugnan por ser el mecanismo de facto de distribución de aplicaciones en Java. Una caché local de artefactos actúa como la primera fuente para sincronizar la salida de los proyectos a un sistema local.
}

En esta seccion ...

\chapter{Trabajos futuros}

\chapter{Conclusiones}
Esta implementación multihilo, se ve limitada a la cantidad de núcleos de procesamiento de una solo m\'maquina, pero si utilizamos la integración con Had


\chapter{Anexo I. C\'odigo fuente}

\bibliografia{bibliografia}

\end{document}