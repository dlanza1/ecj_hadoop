\documentclass{memoriaPFC}

\title{Integración de una herramienta de computo evolutivo y una de procesamiento masivo de información}

\autores{Daniel Lanza Garcia}{}

\director{Francisco Fernandez de Vega}
\directora{}
\codirector{Francisco Chavez de la O}
\codirectora{}

\date{Junio 2015}

\prologo{
En las primeras l\'ineas que describen este proyecto fin de grado, explicamos como se ha realizado la integraci\'on entre dos herramientas bien conocidas por la comunidad, ECJ y Hadoop.

El hecho de mi inter\'es por tecnolog\'ias Big Data y el estar trabajando con investigadores inmersos en la computaci\'on evolutiva, provoc\'o la idea de la uni\'on de ambos campos de conocimiento. Hemos compartido muchas conversaciones donde intent\'abamos refinar la idea, eligiendo las herramientas a integrar, el modo en el que lo hacíamos, mejoras, cambios, etc. Estas conversaciones consegu\'ian mantenernos cada vez m\'as inmersos e inquietos con el proyecto.

Las estancias que hemos realizado con otros grupos de investigaci\'on, en este caso Mexicanos, o algunos de los estudiantes que han venido al centro, han aportado valor al proyecto. Las ideas propuestas por las personas con las que se ha compartido la idea han sido tenidas muy en cuenta. Deducimos fácilmente como la colaboraci\'on entre diferentes grupos o personas no puede m\'as que aportar buenas ideas a un proyecto.

Aunque no alcanzados todos los objetivos deseados, ya que no dejan de surgir ideas, los resultados obtenidos sugieren que el tiempo dedicado ha merecido la pena.
}
\agradecimientos{
Quiero dar mi mas sincero agradecimiento a Don Francisco Fern\'andez de Vega, por haberme dirigido este proyecto de fin de grado. Gracias por su paciencia y por las horas que ha dedicado a este proyecto d\'andole vuelta a ideas o simplemente pensando como podr\'ia quedar mejor la idea final. Sin olvidar su incondicional apoyo a mi trabajo, intentando siempre impulsar mi carrera tan alto como sea posible.

Tambi\'en he de agradecer a Don Francisco Ch\'avez de la O, codirector de \'este proyecto, sin su voluntad y ganas de trabajar no se hubiera podido gestar la idea final que ha surgido. Agradecer las atenciones, el tiempo que ha perdido conmigo en el despliegue de las herramientas y sobre todo por su apoyo t\'ecnico y moral durante nuestra estancia Mexicana.

A los miembros del laboratorio EvoVisi\'on de CICESE (Ensenada, M\'exico), en especial al Doctor Gustavo Olague y a algunos de sus estudiantes, Eddie y Daniel. Durante la estancia en su laboratorio fuimos especialmente acogidos, all\'i se desarroll\'o parte del proyecto y ellos tuvieron especial inter\'es proponiendo ideas.

A los miembros del instituto tecnol\'ogico de Tijuana (M\'exico), teniendo especial atenci\'on con el Doctor Leonardo Trujillo y algunos de sus alumnos, Enrique y Yuliana, que consiguieron hacer de nuestra estancia una experiencia muy placentera y productiva, mostrando su interés por el proyecto y a\~nadi\'endole valor.

A mi jefe y compa\~neros de trabajo en el CERN, los cuales han sido suficientemente flexibles para que pueda dedicarle alguna que otra hora de trabajo al proyecto, sin duda sin este tiempo no podr\'ia haberse llevado a cabo.

A C\'esar Benavides, sin el cual no podr\'ia haberse realizado la integraci\'on del proyecto de reconocimiento facial. Agradecerle el haber hecho su implementaci\'on compatible con las necesidades del proyecto y el tiempo perdido en nuestras continuas reuniones virtuales.

En general, a todos aquellos profesores y alumnos del Centro Universitario de M\'erida que de una u otra forma me han ayudado a realizar este proyecto, que aunque no les mencione de forma expl\'icita, no les puedo negar un sincero agradecimiento.

Tambi\'en agradecer a mis amigos Rafa y Carlos, los cuales me han aguant\'ado durante buena parte de la realizaci\'on del proyecto ayudando de una manera u otra en todo lo que estuviera en sus manos.

Finalmente me gustar\'ia agradecer especialmente el apoyo que me han dado mis padres, Ana y Liborio, los cuales me animan con todo lo que me propongo y como no pod\'ia ser de otra manera tambi\'en he tenido su motivaci\'on y empuje para este proyecto, preocup\'andose de su evoluci\'on en todo momento.

A todos, muchas gracias.
}

\begin{document}

\frontmatter
\hacerportada
\hacercontraportada
\newpage{\ }
\thispagestyle{empty} 
\hacerprologo
\haceragradecimientos
\setcounter{secnumdepth}{2}
\setcounter{tocdepth}{2}
\tableofcontents
\listoffigures
\listoftables

\mainmatter

\chapter{Introducci\'on}
	\section{Motivaciones}
		Muchos problemas computacionales de diferente naturaleza se intentan afrontar haciendo uso de modelos computacionales tradicionales obteniendo no muy buenos resultados, la computación evolutiva aporta un enfoque bioinspirado que en muchos casos consigue proporcionar resultados m\'as que aceptables. Es por esto que su popularidad est\'a en aumento, siendo aplicada esta metodolog\'ia a problemas de muy diversa \'indole.

Con el paso de los a\~nos y la evoluci\'on de los sistemas computacionales, la computaci\'on evolutiva se ha intentado aplicar a la resoluci\'on de problemas cada vez m\'as complejos, lo cual en la mayor\'ia de los casos, suele conllevar el uso de m\'as recursos como capacidad de c\'omputo, memoria y tiempo. Esto hace que se busquen soluciones para mejorar el uso de estos recursos.

Numerosas investigaciones se han llevado a cabo con el fin de minimizar el uso de uno de los recursos m\'as importantes mencionados anteriormente, el tiempo. Se han dise\~nado algoritmos en este modelo cuya ejecuci\'on puede tomar a\~nos y que por lo tanto su tiempo de ejecuci\'on es impracticable. Varias metodolog\'ias se han aplicado para reducir el tiempo de ejecuci\'on, una de ellas es la paralelizaci\'on de parte del proceso, esto puede llevarse a cabo con los procesadores de \'ultima generaci\'on los cuales poseen varios n\'ucleos de procesamiento o tambi\'en se puede conseguir con el uso de varios computadores conectados en red.

Este trabajo plantea una soluci\'on para utilizar los recursos disponibles de forma eficiente y as\'i reducir los tiempos de ejecuci\'on, la soluci\'on que se describir\'a hace uso de una herramienta que explota ambas metodolog\'ias, ejecuci\'on paralela en procesadores multi-n\'ucleo y uso de numerosas computadoras.
	\section{Objetivos}
	\section{Recursos empleados}
	\section{Organizaci\'on del documento}

\chapter{An\'alisis del sistema}
	\section{Introducci\'on}
	\section{Algoritmos evolutivos}
		Existen problemas computacionales que no pueden ser resueltos con técnicas tradicionales, o porque no existe una que pueda proporcionar un resultado aceptable o porque la técnica aplicada necesita de un tiempo o recursos de los que no se disponen. Así es el caso en problemas NP donde una búsqueda exhaustiva encontraría la mejor solución pero el tiempo necesario para su ejecución se hace impracticable. Para este tipo de problemas se buscan técnicas que no proporcionan siempre la mejor soluci\'on pero que intentar acercarse lo máximo posible haciendo uso de recursos razonables, a estos problemas se les conoce como problemas de optimizaci\'on.

Se han desarrollado diferentes formas de afrontar estos problemas de optimizaci\'on y una de ellas es la computación evolutiva, este modelo se basa en las teoría de la evolución que Charles Darwin postul\'o. Esta idea de aplicar la teor\'ia Darwiniana de la evolución surgió en los a\~nos 50 y desde entonces han surgidos diferentes corrientes de investigación:

\begin{itemize}
	\item Algoritmos genéticos, donde los individuos de la población son representados por cadenas de bits o números.
	\item Programación evolutiva, una variación de los algoritmos genéticos, donde lo que cambia es la representación de los individuos. En el caso de la Programación evolutiva los individuos son ternas cuyos valores representan estados de un autómata finito. 
	\item Estrategias evolutivas, se diferencia de las demás en que la representación de cada individuo de la población consta de dos tipos de variables: las variables objeto, posibles valores que hacen que la función objetivo alcance el óptimo, y las variables estratégicas, las cuales indican de qué manera las variables objeto son afectadas por la mutación.. 
\end{itemize}

Este modelo por tanto, se basa generalmente en la evolución de una población y la lucha por la supervivencia. En su teoría, Darwin dictamin\'o que durante muchas generaciones, la variación, la selección natural y la herencia dan forma a las especies con el fin de satisfacer las demandas del entorno, con la misma idea pero con el fin de satisfacer una buena solución al problema que se plantee, surge la computación evolutiva. Podemos observar entonces, algunos elementos importantes como son:

\begin{itemize}
	\item La población de individuos, donde cada una de ellos representa directa o indirectamente una solución al problema.
	\item Aptitud de los individuos, atributo que describe cuanto de cerca est\'a este individuo (solución) de la solución \'optima. 
	\item Procedimientos de sección, es la estrateg\'ia a seguir para elegir los progenitores de la siguiente genraci\'on. \'Esta normalmente elige a los individuos mas apto pero existen otras muchas técnicas.
	\item Procedimiento de transformaci\'on, se lleva a cabo sobre los individuos seleccionados y puede consistir en la combinación de varios individuos o en la mutación (cambios normalmente aleatorios en el individuo).
\end{itemize}

Para llevar a cabo la implementación en computadoras, se ha dividido el problema en diferentes fases y procedimientos, los cuales se ejecutan con un orden determinado, describimos a continuación de forma general como se lleva a cabo la resolución de problemas utilizando este modelo.

\begin{enumerate}
	\item Inicializaci\'on, en esta primera fase se genera la población inicial, normalmente se genera una cantidad de individuos que es configurada y cada uno de ellos se genera de manera aleatoria, siempre respetando las restricciones que el problema imponga a la solución.
	\item Evaluaci\'on, se calcula la aptitud de cada uno de los individuos de la población para poder determinar posteriormente cuales son m\'as aptos.
	\item Las fases que siguen a continuación se repiten hasta que se cumpla una de las siguientes dos condiciones: que se encuentre la solución \'optima o que se alcance un l\'imite impuesto por el programador como un n\'umero de generaciones máximo o un tiempo máximo.
	\begin{enumerate}
		\item Selección, siguiendo la estrategia de selección de progenitores elegida, se eligen individuos de la población. Normalmente los que sean m\'as aptos.
		\item Procreaci\'on, utilizando los individuos seleccionados, se combinan para generar nuevos individuos, y con ellos una nueva población (generación).
		\item Mutaci\'on, a un porcentaje de los individuos recién generados se les aplica una modificación aleatoria.
		\item Evaluaci\'on, todos los individuos de la nueva poblaci\'on son evaluados.
	\end{enumerate}
\end{enumerate}

Con el fin de entender mejor este proceso, se muestra una imagen \ver{fases-evolutivo} donde se observan cada una de las etapas descritas anteriormente.

\figuraSinMarco{0.7}{imagenes/fases-evolucion}{Fases del proceso evolutivo}{fases-evolutivo}{}

Varias herramientas han surgido en la comunidad para ayudar a la investigación de este modelo, en diferentes lenguajes de programación y plataformas. En nuestro caso hemos elegido ECJ que es un framework bien conocido por la comunidad, implementado en el lenguaje de programación Java y que posee una flexibilidad importante para la ejecución de problemas de muy distinta naturaleza. M\'as adelante \verapartado{desarrollo-ecj} se describe con m\'as detalle esta herramienta, explicando su funcionamiento e implementación.

\subsection{Paralelizaci\'on\label{analisis-evolutivos-paralelizacion}}

Como hemos comentado anteriormente \verapartado{motivaciones} este proyecto surge de la necesidad de optimizar el uso de recursos cuando se intentan resolver problemas complejos con este modelo. Con este fin, y con la posibilidad de paralizar el procesamiento de algunas de las partes del proceso, surge la viabilidad de este proyecto.

Varias partes del proceso evolutivo pueden ser paralizadas, pero no todas merecen el esfuerzo ya que el coste en algunas de ellas es mínimo. Una de las fases que suele conllevar un coste computacional alto y que su paralelizaci\'on en la mayoría de problemas es sencilla, es la fase de evaluaci\'on de individuos.

Se han utilizado diferentes técnicas para este prop\'osito //TODO(incluir referencias), una de ellas es la ejecución de la evaluación de individuos haciendo uso de procesadores multin\'ucleo/multihilo. Esta t\'ecnica consigue buenos resultados pero se limita a las capacidades del procesador que posea la computadora donde se ejecute. Otro intento para llevar a cabo la paralelizaci\'on del proceso ha sido la ejecución en diferentes m\'aquinas, las cuales se conectan haciendo uso de una red. Este planteamiento requiere de una implementación m\'as compleja y no suele explotar todos los recursos de forma eficiente, adema\'as de que algunas soluciones planteadas carecen de la escalabilidad deseada. Tambi\'en han surgido implementaciones que hacen uso de ambas t\'ecnicas, esta solución suele ser la m\'as apropiada ya que hace un uso m\'as eficiente del hardware disponible, aunque requiere de una implementación a\'un m\'as compleja.

Con el fin de llevar a cabo la paralelizaci\'on, han surgido diferentes modelos paralelos, algunos de los cuales se describen a continuación:

\begin{itemize}
	\item Modelo maestro-esclavo, se mantiene una población donde la evaluación del fitness y/o la aplicación de los operadores genéticos se hace en paralelo. Se implementan procesos maestro-esclavo, donde el maestro almacena la población y los esclavos solicitan parte de la población al proceso maesto para evaluarla.
	\item Modelo de Islas, la población de individuos se divide en subpoblaciones que evolucionan independientemente. Ocasionalmente, se producen migraciones entre ellas, permitiéndoles intercambiar información genética. Con la utilización de la migración, este modelo puede explotar las diferencias en las subpoblaciones; esta variación representa una fuente de diversidad genética. Sin embargo, si un número de individuos emigran en cada generación, ocurre una mezcla global y se eliminan las diferencias locales, y si la migración es infrecuente, es probable que se produzca convergencia prematura en las subpoblaciones.
	\item Modelo celular, se coloca cada individuo en una matriz, donde cada uno sólo podrá buscar reproducirse con los individuos que tenga a su alrededor escogiendo al azar o al mejor adaptado. El descendiente pasar\'a a ocupar una posición cercana. No hay islas en este modelo, pero hay efectos potenciales similares. Asumiendo que el cruce esta restringido a individuos adyacentes, dos individuos separados por 20 espacios están tan aislados como si estuvieran en dos islas, este tipo de separación es conocido como aislamiento por distancia. Luego de la primera evaluación, los individuos están todavía distribuidos al azar sobre la matriz. Posteriormente, empiezan a emerger zonas como cromosomas y adaptaciones semejantes. La reproducción y selección local crea tendencias evolutivas aisladas, luego de varias generaciones, la competencia local resultara en grupos m\'as grandes de individuos semejantes.
\end{itemize}

La implementación del paralelismo de estos y otros modelos han hecho uso de tecnologías que permitan la comunicación y distribución de los diferentes procesos que llevan a cabo el proceso evolutivo, algunas de ellas se describen a continuación:

\begin{itemize}
	\item MPI, es un estándar que define la sintaxis y la semántica de las funciones contenidas en una biblioteca de paso de mensajes diseñada para ser usada en programas que exploten la existencia de múltiples procesadores.
	\item PVM, es una biblioteca para el cómputo paralelo en un sistema distribuido de computadoras. Está diseñado para permitir que una red de computadoras heterogénea comparta sus recursos de cómputo (como el procesador y la memoria RAM) con el fin de aprovechar esto para disminuir el tiempo de ejecución de un programa al distribuir la carga de trabajo en varias computadoras.
	\item BOINC, una plataforma de computación voluntaria de propósito general para proyectos de computación distribuida, que permite compartir los recursos de las computadoras de sus contribuyentes con otros proyectos. 
\end{itemize}

El inter\'es por explorar el uso de otras tecnologías para la paralelizaci\'on de los procesos evolutivos nos ha llevado a la integración de una herramienta de c\'omputo evolutivo con una herramienta que tiene como principal propósito el procesamiento masivo de información, algo que conlleva la distribución y paralelizaci\'on de multiples tareas a lo largo de amplios clusters de computadores.





















	\section{Procesamiento masivo de informaci\'on}
		Vivimos en el momento m\'as \'algido en la genraci\'on de informaci\'on, nunca antes había existido plataformas que generaran la cantidad de información que se genera hoy. El hecho de que del análisis de grandes cantidades de información se puedan extraer valiosos datos como estrategias de negocio, hacen que numerosas empresas y organizaciones almacenen cantidades ingentes de información para poder sacarle el máximo partido.

La generaci\'on de informaci\'on se est\'a produciendo en \'ambitos muy dispares, estos pueden ser redes sociales que mantienen millones de usuarios, grandes empresas con muchos clientes, laboratorios de f\'isica con redes de millones sensores y muchos otros ejemplos que podríamos mencionar. Todos ellos queriéndole sacan el máximo valor a la información que recaban.

\subsubsection{Computaci\'on distribuida}

Cuando hablamos de cantidades de información\'on del orden de terabytes o petabytes no podemos pensar en otra cosa que no sea computación distribuida. Para una sola m\'aquina, el tiempo que conllevar\'ia procesar esas cantidades de datos podr\'ian ser del orden de a\~nos.

Por tanto, la computaci\'on distribuida es la \'unica soluci\'on con el hardware que hoy en d\'ia manejamos. Esta soluci\'on implica el uso de m\'ultiples computadores conectados a la red, cada una de las cuales tiene su propio procesador, arquitectura, etc, con lo que pueden ser totalmente heterog\'eneas, en contraposici\'on a la computaci\'on paralela, que consiste en utilizar m\'as de un hilo de procesamiento simult\'aneamente para ejecutar un \'unico programa. Idealmente, el procesamiento paralelo permite que un programa se ejecute m\'as r\'apido, en la pr\'actica, suele ser difi\'icil dividir un programa de forma que CPU separadas ejecuten diferentes porciones del programa sin ninguna interacci\'on.

\subsection{Modelo computacional: Map/Reduce}

En este \'ambito, surge la necesidad de dise\~nar herramientas que puedan no solo mantener esta informaci\'on, si no también que tengan la capacidad de analizarla y extraer el valor que se desea de una forma distribuida y con un modelo sencillo. 

\textit{Google}, el buscador de internet m\'as utilizado en el mundo, ha hecho frente ha este problema antes que nadie ya que desde hace a\~nos maneja cantidades de información realmente grandes, es por esto que sus investigaciones y experiencia son avanzadas. Varios a\~nos atrás hicieron p\'ublica \cite{paper-mapreduce} una solución para el an\'alisis de grandes cantidad de datos de forma distribuida y con un modelo que da soluci\'on a la complejidad de dividir el problema para poder paralelizarlo, ha este modelo se le conoce como Map/Reduce. Esta publicación ha dado pie a que se implemente una herramienta que se conoce con el nombre de Hadoop \cite{hadoop}, la cual se describe con m\'as detalle en un cap\'itulo posterior \verapartado{desarrollo-hadoop}. La creaci\'on de esta herramienta y el hecho de que se hayan obtenidos buenos resultados de ella, ha provocado que surjan otras muchas herramientas a su alrededor, las cuales ayudan a diferentes tareas como el volcado de información (Sqoop), bases de datos para consulta (Impala), gestion de flujos de datos (Flume) y otras muchas.

La propuesta de computación distribuida que se hace en esta publicación, hace uso de un modelo computacional anteriormente conocido en la programación funcional. Este modelo se basa en aplicar dos funciones básicas conocidas como Map (mapeo) y Reduce (reducci\'on). La función de mapeo consiste en aplicar una transformación o procedimiento a todos los datos, obteniendo así la entrada de la siguiente fase, reducción, la cual consiste en ''resumir", aplicando la misma función a diferentes partes de la salida de la fase de mapeo, obteniendo finalmente la salida deseada. Numerosas fases de mapeo y reducción pueden ser concatenadas en el orden que se quiera con el fin de producir el resultado esperado. Este modelo es el implementado en Hadoop pero con algunas peculiaridades\verapartado{desarrollo-hadoop-implementacion-modelo}.


\chapter{Desarrollo del proyecto}
	\section{Estudio de las herramientas a integrar}
		\subsection{ECJ como herramienta de c\'omputo evolutivo}
		\subsection{Hadoop como herramienta de procesamiento masivo de informaci\'on}
	\section{Implementaci\'on}

\chapter{Manual de usuario}

\chapter{Trabajos futuros}

\chapter{Conclusiones}

\chapter{Anexo I. Instalacion de herramientas, compilacion y ejecutables}

\chapter{Anexo II. Herramientas usadas durante el PFC}

\chapter{Anexo III. Codigo fuente}

\bibliografia{bibliogra}

\end{document}
