Esta sección tiene como objetivo explicar el procedimiento para importar el proyecto en un entorno de desarrollo, esto no es necesario para su ejecución por lo que no estén interesados en modificar/ampliar el proyecto pueden continuar la lectura en la sección siguiente (Compilación).

En el apartado anterior se ha explicado como obtener el proyecto, pero lo obtenido es básicamente los ficheros de código fuente, no se incluyen proyectos de entornos de desarrollo ni ejecutables.

Si acabamos de descargar/clonar el proyecto, no lo tendremos importado en nuestro entorno, para llevar a cabo esta operación tan solo debemos dirigirnos al menu de importación de proyectos, solucionar que es un proyecto Maven y seleccionar el directorio que contiene el proyecto. 

Otra opción, teniendo la herramienta Maven instalada en el sistema \cite{instalacion_maven}, es generar primero el proyecto del entorno de desarrollo y posteriormente importarlo como un proyecto normal. Para generar el proyecto del entorno de desarrollo debemos abrir una consola, dirigirnos al directorio donde se sitúa el proyecto y ejecutar el comando Maven correspondiente a nuestro entorno, por ejemplo para Eclipse debemos ejecutar el siguiente comando:

\mostrarconsola{
	[usu@host repo]\$ mvn eclipse:eclipse
}

Esto nos generar\'a el proyecto Eclipse y podremos importarlo como cualquier otro proyecto.