La computaci\'on evolutiva est\'a adquiriendo cada vez mas importancia a lo largo de los a\~nos, sus cada vez mas aplicaciones en sistemas de diferente naturaleza la hacen imprescindible para la resoluci\'on de problemas, que haciendo uso de otro modelo computacional, no podr\'ian ser resueltos.

Con el paso de los a\~nos y la evoluci\'on de los sistemas computacionales, la computaci\'on evolutiva se ha intentado aplicar para la resoluci\'on de problemas cada vez mas complejos, lo cual en la mayor\'ia de los casos, suele conllevar el uso de m\'as recursos como capacidad de c\'omputo, memoria y tiempo. Esto hace que se busquen soluciones para mejorar el uso de estos recursos.

Numerosas investigaciones se han llevado a cabo con el fin de minimizar el uso de uno de los recursos mas importantes mencionados anteriormente, el tiempo, se han dise\~nado algoritmos en este modelo cuya ejecuci\'on puede tomar a\~nos y que por lo tanto su tiempo de ejecuci\'on es impracticable. Varias metodolog\'ias se han aplicado para reducir el tiempo de ejecuci\'on, una de ellas es la paralelizaci\'on de parte del proceso, esto puede llevarse a cabo con los procesadores de \'ultima generaci\'on los cuales poseen varios n\'ucleos de procesamiento o tambi\'en se puede conseguir con el uso de varios computadores conectados en red.

Este trabajo plantea una soluci\'on para utilizar los recursos disponibles de forma eficiente y as\'i reducir los tiempos de ejecuci\'on, la soluci\'on que se describir\'a hace uso de una herramienta que explota ambas metodolog\'ias, ejecuci\'on paralela en procesadores multi-n\'ucleo y uso de numerosas computadoras.