\prologo{
En las primeras lineas que describen este proyecto fin de grado, explicamos como se ha realizado la integraci\'on entre dos herramientas bien conocidas por la comunidad, ECJ y Hadoop. Otras integraciones con estas herramientas se han realizado pero no concretamente entre estas dos.

El hecho de mi inter\'es por tecnolog\'ias Big Data y el estar trabajando con investigadores inmersos en la computaci\'on evolutiva provoc\'o la idea de la uni\'on de ambos campos de conocimiento. Hemos compartido muchas conversaciones donde intent\'abamos refinar la idea, eligiendo las herramientas a integrar, el modo en el que lo hacíamos, mejoras, cambios, etc. Estas conversaciones consegu\'ian mantenernos cada vez mas inmersos e inquietos con el proyecto.

Las estancias que hemos realizado con otros grupos de investigaci\'on en este caso Mexicanos o algunos de los estudiantes que han venido al centro han aportado valor al proyecto, estos han propuesto ideas de las cuales algunas han sido implementadas. Deducimos fácilmente como la colaboraci\'on entre diferentes grupos o personas no puede mas que aportar buenas ideas a un proyecto.

Aunque no alcanzados todos los objetivos deseados, ya que no dejan de surgir ideas, los resultados obtenidos sugieren que el tiempo dedicado ha merecido la pena.
}
