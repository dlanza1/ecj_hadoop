El documento que se presenta ha sido dividido en siete cap\'itulos. El primero de ellos consiste en una introducci\'on \verapartado{introduccion} donde se explican las motivaciones que han llevado a la realizaci\'on del proyecto, sus objetivos y las herramientas y recursos utilizados.

En el segundo cap\'itulo \verapartado{analisis}, nos sumergimos en los dos campos de conocimiento que este trabajo contempla, algoritmos evolutivos y procesamiento masivo de informaci\'on, describiendolos con detalle.

Pasamos en el siguiente cap\'itulo \verapartado{integracion} al estudio de las herramientas que se van a integrar, y explicar c\'omo se realiza la integraci\'on introduciendo algunos diagramas donde se entienda mejor la filosofía de la solución. Al final de este cap\'itulo se describen algunas mejoras introducidas y se muestran los resultados obtenidos de las ejecuciones llevadas a cabo utilizando la integraci\'on y sin utilizarla.

Un caso especial en el que desarrollamos una integraci\'on diferente se ve en el cap\'itulo cuarto \verapartado{facerecognition}, donde se desarrolla una implementaci\'on que utiliza un modelo diferente al del cap\'itulo anterior. En este cap\'itulo integramos un software de reconocimiento facial con ECJ y con Hadoop, haciendo uso de una librer\'ia de procesamiento de im\'agenes conocida por el nombre de OpenCV.

Con el prop\'osito de ense\~nar como utilizar la implementaci\'on realizada se desarrolla un manual de usuario \verapartado{manual-usuario}. Se explica paso a paso como obtener la implementaci\'on, compilarla, configurarla y ejecutarla de manera sencilla para que un usuario sin muchos conocimientos de Hadoop y ECJ pueda utilizarla.

Durante la realizaci\'on del proyecto, y una vez finalizado, surgen ideas para mejorar o ampliar el proyecto,  durante uno de los cap\'itulos \verapartado{trabajo-futuro} se describen estas ideas.

Como \'ultimo cap\'itulo \verapartado{conclusiones} se comentan las conclusiones a las que se ha llegado tras el \'analisis de los resultados obtenidos en los diversos problemas planteados, discutiendo cuando el uso de esta soluci\'on es apropiado y cuando no.

Para finalizar el documento se ha incluido una bibliografía donde se enumeran las diferentes fuentes mencionadas. Se incluyen desde art\'iculos hasta p\'aginas webs de tutoriales pasando por algunas referencias a im\'agenes utilizadas.